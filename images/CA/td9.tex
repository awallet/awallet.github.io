\documentclass[11pt]{exam}
%%%%%%%%%%%%%%%%%%%%%%%%%%%%%%%%
\noprintanswers % pour enlever les réponses
%\printanswers

\unframedsolutions
\SolutionEmphasis{\itshape\small}
\renewcommand{\solutiontitle}{\noindent\textbf{A: }}
%%%%%%%%%%%%%%%%%%%%%%%%%%%%%%%%



%\usepackage[margin=0.73in]{geometry}
%\usepackage[top=1in, bottom=1in, left=1in, right=1in]{geometry}

%\usepackage{fullpage}


\usepackage{hyperref}
\usepackage{appendix}
\usepackage{enumerate}


\usepackage{times,graphicx,epsfig,amsmath,latexsym,amssymb,verbatim}%,revsymb}
\usepackage{algorithmicx, enumitem, algpseudocode, algorithm, caption}


%%%%%%%%%%%%%%%%%%%%%
% Handling comments and versions %%%
%%%%%%%%%%%%%%%%%%%%%
\newcommand{\extra}[1]{}

\renewcommand{\comment}[1]{\texttt{[#1]}}

\newcommand*{\ScProd}[2]{\ensuremath{\langle#1\!\mathbin{,}\!#2\rangle}} %Scalar Product


%%%%%%%%%%%%%%%%%%%%%%%%%%%
%% THEOREMS
%%%%%%%%%%%%%%%%%%%%%%%%%%%

\usepackage{amsmath,amssymb,amsfonts}
\usepackage{amsthm}

\newtheorem{theorem}{Theorem}[section]
\newtheorem{axiom}[theorem]{Axiom}
\newtheorem{conclusion}[theorem]{Conclusion}
\newtheorem{condition}[theorem]{Condition}
\newtheorem{conjecture}[theorem]{Conjecture}
\newtheorem{corollary}[theorem]{Corollary}
\newtheorem{criterion}[theorem]{Criterion}
\newtheorem{definition}[theorem]{Definition}
\newtheorem{lemma}[theorem]{Lemma}
\newtheorem{notation}[theorem]{Notation}
\newtheorem{proposition}[theorem]{Proposition}


\theoremstyle{definition}
\newtheorem{problem}{Problem}


\newcommand{\nc}{\newcommand}
\nc{\eps}{\varepsilon}
\nc{\RR}{{{\mathbb R}}}
\nc{\CC}{{{\mathbb C}}}
\nc{\FF}{{{\mathbb F}}}
\nc{\NN}{{{\mathbb N}}}
\nc{\ZZ}{{{\mathbb Z}}}
\nc{\PP}{{{\mathbb P}}}
\nc{\QQ}{{{\mathbb Q}}}
\nc{\UU}{{{\mathbb U}}}
\nc{\OO}{{{\mathbb O}}}
\nc{\EE}{{{\mathbb E}}}


\newcommand{\field}{\mathbb{K}}
\nc{\K}{K}
\newcommand{\M}{\mathcal{M}}
\newcommand{\val}{\operatorname{val}}
\newcommand{\bigO}{\mathcal{O}}
\DeclareMathOperator{\Span}{Span}
\newcommand{\vc}[1]{\mathbf{#1}}

\pretolerance=1000

%%%%%%%%%%%%%%%%%%%%%%%%%%%%%%%%
%%%%%%%%%%%%%%%%%%%%%%%%%%%%%%%%
%% DOCUMENT STARTS
%%%%%%%%%%%%%%%%%%%%%%%%%%%%%%%%
%%%%%%%%%%%%%%%%%%%%%%%%%%%%%%%%
\usepackage{tikz}
\usetikzlibrary{automata}
\DeclareMathOperator{\Vol}{Vol}

\begin{document}

{\noindent
   \textsc{ENS Lyon --  M1 -- Computer Algebra}
   \hfill {E.Kirshanova / A.Wallet // 2017--2018\\
  }
  \hrule
% Titre de la feuille
  \begin{center}
    {\Large\textbf{
   \textsc{Tutorial 9}
    } } 
  \end{center}
  \hrule \vspace{5mm}

\thispagestyle{empty}

\vspace{0.2cm}

%\section{SPEEDING FERMAT’S FACTORING METHOD}
% REMARK: should go after Lehman's


\section{Modular roots and factoring, again}

Last week, we saw Tonelli-Shanks algorithm to compute square roots modulo an odd prime $p$ in $O(\log^3 p)$. The first step of this exercise is to design an algorithm to compute square roots modulo $p^v$, for some $v\geq 2$ and odd prime $p$.
  

\begin{questions}
  \question Let $x \in (\ZZ/p^v\ZZ)^\times$. Show that $x^2 \equiv 1~[p^v]$ if and only if $x \equiv \pm 1~[p^v]$. Let $\varphi$ be the Euler totient function. Deduce a generalization of Euler's criterion:
  $$x^{\varphi(p^v)/2} \equiv \begin{cases} 1~\text{if}~x~\text{is a square}\bmod p^2,\\
    -1~\text{if}~x~\text{is not a square}\bmod p^2.\end{cases}$$
  \begin{solution}
    Let $x^2 \equiv 1~[p^2]$. Therefore $p^2$ divides $x^2-1=(x+1)(x-1)$, so in particular $p$ divides one of these factors. It divides exactly one of them, say $x+1$, else it would divide their difference $x-1 -x +1=2$, which contradicts the definition of $p$. In other words, $p$ is coprime to $x-1$. As $p^2$ divides $x^2-1$, this implies that $p^2$ divides $x+1$, hence the result.

    This also means that $X^2-1$ has exactly two roots modulo $p^2$, or in other words that the kernel of the application $x\mapsto x^2$ defined over $(\ZZ/p^v\ZZ)^\times$ has $2$ elements and that the ring contains $\varphi(p^v)/2$ square elements. Using the same argument as in the proof of Euler's criterion, we obtain the generalized criterion.
  \end{solution}

  \question Show that an integer $x$ coprime to $p$ is a square modulo $p$ if and only if it is a square modulo $p^v$.
  
  \begin{solution}
    If $x$ is a square modulo $p^v$, we may write $x=a^2+kp^v=a^2+k'p$ for some $k$ and $k'=kp^{v-1}$, hence $x$ is a square modulo $p$. Assume $x$ is not a square modulo $p^v$, so that our crierion gives $x^{\varphi(p^v)/2}\equiv -1~[p^v]$, which also means $x^{\varphi(p^v)/2}\equiv -1~[p]$, with the same argument we just used. We readily compute $\varphi(p^v)=(p-1)p^{v-1}$, and rewrite $x^{\varphi(p^v)/2} = (x^{(p-1)/2})^{p^{v-1}}$. Repeated applications of Fermat's (or Lagrange's) theorem gives $(x^{(p-1)/2})^{p^{v-1}} \equiv x^{(p-1)/2} \equiv -1~[p]$, and we conclude with Euler's criterion.

  \end{solution}

  We will now use a Hensel-like strategy: assume we know $y \in (\ZZ/p^{v-1}\ZZ)^\times$ a square root of $x$ modulo $p^{v-1}$.

  \question  If $z$ is a square root of $x$ modulo $p^{v}$, show that $z\equiv \pm y~[p^{v-1}]$.

  \begin{solution}
    By definition, we may find an integer $k$ such that $z^2 = x+kp^v = x+k'p^{v-1}$. This also means that $z^2 \equiv y^2~[p^{v-1}]$. Since $y$ is invertible modulo $p^{v-1}$, we can write $(zy^{-1})^2 \equiv 1~[p^{v-1}]$, and the result follows from question $1$.

  \end{solution}

  \question We keep assuming $z$ is a square root of $x$ modulo $p^v$. Show that there is an integer $k$ such that $x-y^2 \equiv \pm 2ykp^{v-1}~[p^{v}]$. Then use that $2y \in (\ZZ/p\ZZ)^\times$ and the previous question to give an expression for $z$.

  \begin{solution}
    We know that $z=\pm y+kp^{v-1}$ for some integer $k$. Then by definition, $x \equiv z^2 \equiv y^2\pm 2ykp^{v-1}~[p^v]$, thus the first claim. Assume we have $z=y+kp^{v-1}$. We know that $p$ is odd and that $y$ is invertible modulo $p^{v-1}$, hence coprime to $p$, so $2y$ is indeed invertible modulo $p$. If $a$ is its inverse, we can write $2ya = 1 +lp$ for some integer $l$, and we obtain $a(x-y^2) \equiv kp^{v-1} [p^v]$ as well. The natural candidate for $z$ is therefore $z:=y+a(x-y^2)$, which we can compute from the knowledge of $y$. The last step is to check that $z$ is indeed a square root of $x$ modulo $p^v$:
    \begin{align*}z^2 = y^2+2a(x-y^2)+a^2(x-y^2)^2 \equiv&~y^2 +2a(x-y^2)~[p^v]\\
      \equiv&~y^2 + 2ykp^{v-1}~[p^v]\\
      \equiv&~x~[p^v].
    \end{align*}
    The other expression (when $z=-y+kp$ is obtained similarly).
    
  \end{solution}

  \question Deduce an algorithm to compute square roots modulo $p^v$ and give its complexity.

  \begin{solution}
    First compute a square root of $x$ modulo $p$ using Tonelli-Shanks. This costs $O(\log^3 p)$ (or $O(\log^2 p)$ using quadratic reciprocity law and smarter computations). Compute the inverse of $2y$ modulo $p$ using Extended Euclidean Algorithm. Then for $i$ from $1$ to $v$, $y_{i+1} \gets y_i + a(x-y_i^2) \bmod p^{i+1}$, which can be computed in $O(\log^2 p)$ operations (TO CHECK).

  \end{solution}
  
  The second step is to show that factoring an integer $N$ and computing square roots modulo $N$ are equivalent problems.
  
  \question Write $N=p_1^{v_1}\dotsc p_r^{v_r}$, for some distinct primes $p_1, \dotsc, p_r$. Show that $x$ is a square modulo $N$ if and only if $x^{(p_i-1)/2}\equiv 1~[p_i]$. Also show that a square modulo $N$ has $2^r$ square roots.

  \begin{solution}
    If $x\equiv y^2~[N]$, then in particular it is a square modulo each $p_i^{v_i}$. Question $2$ shows that it is a square modulo each $p_i$, and we conclude by Euler's criterion. Conversely, if $x$ is a square modulo each $p_i$, it is a square modulo $p_i^{v_i}$ by Question 2, so there exists elements $y_i \in (\ZZ/p_i^{v_i}\ZZ)^\times$ such that $x\equiv y_i^2~[p_i^{v_i}]$. The moduli are pairwise coprime, so using CRT, we may find a solution to this systems of congruences, that is to say, an square root of $x$ modulo $N$.

    Modulo each $p_i^{v_i}$, we know that $x$ has exactly two square roots. We conclude using the CRT in its ``algebraic version'': we have a ring isomorphism $\ZZ/N\ZZ \simeq \prod_{i=1}^r \ZZ/p_i^{v_i}\ZZ$.
  \end{solution}
    \question Assume we have an algorithm {\sc A} that computes square roots in $\ZZ/N\ZZ$ in time $\tau$. Show that we can find a factor of $N$ in expected time $O(\tau(1-2^{1-d}))$. Conclude on the respective difficulty of these problems.

    \begin{solution}
      The algorithm samples $y$ uniformly at random in $\ZZ/N\ZZ$ and computes $y^2$. Then it runs {\sc A} to get $z$, which is one among the $2^r$ square roots of $y$. In particular, we have $z^2 \equiv y^2~[N]$, or equivalently $(z-y)(z+y)\equiv 0~[N]$: we should find a factor of $N$ by computing gcd($N, z-y$) and gcd($N, z+y$), provided $z\not\equiv \pm y~[p_i^{v_i}]$ for at least one $i$. In other words, the algorithm fails when $z\equiv \pm y~[p_i^{v_i}]$ for all $i$, which happens with probablity $2^{1-r}$. This gives the expected running time. Conversely, if we know the factorization of $N$ and we need to compute a square root of $x$ modulo $N$, then we can just compute a square root modulo each of the prime power factors, and use CRT to recover a square root modulo $N$.
      
    \end{solution}

  
\end{questions}

\section{McKee's factoring algorithm}

An old factoring strategy is to find two different ways to represent $n$ as a sum of two squares. Unfortunately, not any number can be represented as a sum of two squares, and, even if it can, the representation is, in general, hard to find. Nevertheless, there are algorithms that exploit this idea. In this exercise, we develop the factorization algorithm due to McKee that finds a non-trivial factor of $n$ \emph{provably} in time $\mathcal{O}(n^{1/2 + \varepsilon})$ and \emph{heuristically} in time $\mathcal{O}(n^{1/4} + \varepsilon)$.

Suppose $n = pq$ and $2n^{1/4} < p<q$. Let 
\[
	 b = \lceil \sqrt{n} \rceil.
\]
Define
\[
	Q(x,y) = (x+by)^2-ny^2.
\]
The idea is to search for integers $x,y$ s.t.\ 
\begin{align} \label{eq:quad_form}
	Q(x,y) = z^2
\end{align}
for an integer $z$. A triple $(x,y,z)$, as you now show, gives a non-trivial factor of $n$.
\begin{questions}
	\question \label{q:first} Set $r = \lfloor \sqrt{(q/p)} \rfloor$ and 
	\[
		y = 2 \cdot r,  \quad x = r^2p+q - by, \quad z = q-r^2p.
	\]
	Show that 
	\begin{enumerate}
		\item $(x,y,z)$ defined as above gives a solution to \eqref{eq:quad_form}
		\item $2 \leq y \leq n^{1/4}$,
		\item $0 \leq z < 2n^{1/2}$,
		\item $|x|y < 2n^{1/2}$,
		
		\item $\gcd(x+by-z, n)$ gives a non-trivial factor of $n$.
	\end{enumerate}
	\begin{solution}
		The exercise is taken from \url{http://www.ams.org/journals/mcom/1999-68-228/S0025-5718-99-01133-3/S0025-5718-99-01133-3.pdf}.
		
		The fact that $(x,y,z)$ form a solution to \eqref{eq:quad_form} directly follows from substitution.
		
		The second fact is an immediate consequence of bounds on $p,q$: as $q>p$, $q/p>1$ so $y\geq 2$. By definition, we have $q<n^{3/4}/2$, so $q/p \leq n^{3/4}/(4n^{1/4})=n^{1/2}/4$ and the second bound follows. 
		
		For 3., we have $\sqrt{(q/p)}-1 < r \leq \sqrt{q/p}$ and, hence, $0 \leq z  = q-r^2p < q-q+2\sqrt{n}-p < 2 \sqrt{n}$.
		
		For 4., write $b =\sqrt{n}+\delta$ with $0 \leq \delta <1$. Then $(x+by)^2 = x^2+2xy(\sqrt{n}+\delta)+y^2n+2\delta y^2 \sqrt{n} + \delta^2 y^2 = y^2n+z^2$, where the last equality comes from the fact that $(x, y,z)$ is a solution. If $x \geq 0$, by re-arranging the terms we get
		\[
			xy < \frac{z^2}{2(\sqrt{n}+\delta)} < 2 \sqrt{n},
		\]
		where for the last inequality we used the result from 3.
		
		If $x<0$, write $r  = \sqrt{(q/p)} - \eta$ for $\eta \geq 0$. Thus, $x = q - 2p\sqrt{q/p}\eta + p \eta^2 + q - (\sqrt{n}+\delta) \cdot 2( \sqrt{(q/p)} - \eta) = \eta^2p+2\eta \delta - 2 \sqrt{n} \delta / p$. It follows that $|x| < 2 \sqrt{n}/p < n^{1/4}$ and $|x|y < \sqrt{n}<2\sqrt{n}$.
		
		To get 5., from 2., 3., 4., we have  $|x+by \pm z| < | < 2\sqrt{n}/y+(n^{1/2}+1)y +2\sqrt{n} < n^{3/4} + 3n^{1/2}+n^{1/4} < n$ for $n$ large enough; anyway, we do not care about factoring small numbers. Because $(x,y,z)$ is a solution, the case $x+by = z$ implies that $y=0$, contradicting 2. Hence, $\gcd(x+by-z, n)$ gives a non-trivial factor of $n$. 
	\end{solution}
	\item Using the above, give an algorithm that finds a non-trivial factor of $n$ in time $\mathcal{O}(n^{1/2+ \varepsilon}).$
	\begin{solution}
		Using the results 2. and 3. from the above, we search over all even $y < n^{1/4}$ for a corresponding $x < n^{1/2}/y$ s.t. $Q(x,y)$ gives a square. We need $\sum_{i=2}^{n^{1/4}} \sum_{j=1}^{\sqrt{n}/i} 1 = \sqrt{n} \sum_{i=1}^{n^{1/4}} i^{-1} \approx n^{1/2}\cdot \log(n^{1/4})$ evaluations of $Q$, since the second sum is harmonic. The assumption that $p>n^{1/4}$ can be verified by trial divisions in time $\mathcal{O}(n^{1/4})$, and we obtain the complexity.
	\end{solution} 
	\question For an heuristic version of the algorithm we would want to have many solutions to \eqref{eq:quad_form}. Refine the results of question \ref*{q:first} by showing that for an integer $T>1$:
	\begin{enumerate}
		\item $2 < y < n^{1/4} + 2(T-1)$,
		\item $|x|y < T^4 \sqrt{n} $,
		\item $|z| < (T^2 - 1) \sqrt{n}$,
		\item there exist at least $(T-1)$ solutions to \eqref{eq:quad_form} for $x>0$.
	\end{enumerate}
	\begin{solution}
		One can verify by substitution that triples of the form $(x = (r+t)^2p+q-by, y=2\cdot(r+t), z=q - (r+t)^2p )$ for $0\leq t \leq T-1$ give solutions to \eqref{eq:quad_form}. This gives 4. The upper bound on $y$ follows from question 2 and the way we define $y$.
		
		Let $t>0$ (the case $t=0$ is already shown in question 2). Then
		\[
			0 > z = q-(r+t)^2 p \geq q - (\sqrt{q/p}+t)^2 p = -2t \sqrt{n} - t^2p.
		\]
		Taking absolute values and using the fact that $p<\sqrt{n}$, we obtain
		\[
			|z| < 2t \sqrt{n}+t^2p < t\sqrt{n}(2+t) < (T^2-1) \sqrt{n}.
		\]
		This gives 3. For 2., since $t \geq 1$, we have that $x$ is always positive and we use this case in the solution to question 2 to show that
		\[
		0<xy < \frac{z^2}{2 \sqrt{n}} < (T^2-1)^2 \sqrt{n}/2 < T^4 \sqrt{n}.
		\]
	\end{solution}
	Our heuristic assumption states that  the number of solutions to \eqref{eq:quad_form} is large enough so that there exist a solution triple $(x, y,z)$ such that the following two conditions are met: 
	\begin{itemize}
		\item $\gcd(x,y,z)=1$,
		\item  there exist $m$ that divides $z$.  
	\end{itemize}
	\question  Show that under the above assumptions for $x>0$ and $m^2 > 2xy$, there exist $x_0 < m^2$ and $\lambda \in \ZZ$ s.t.
	\[
		0 < \frac{x_0}{m^2} - \frac{\lambda}{y} < \frac{1}{2 y^2}.
	\]
	Note that $\lambda / y$ can be extracted from the continued fraction expansion of $x_0/m^2$.
	\begin{solution}
		The desired $x_0$ is of the form $x_0 = xy^{-1}\bmod m^2$. Indeed, for $x, y$ we have $Q(x,y) = z^2$ and due to our second assumption $Q(x,y) = (x+by)^2 - ny^2 =0 \bmod m^2$. Dividing by $y^2$, gives $Q(x_0, 1) = 0 \bmod m^2$. By definition, we have $x = x_0y - \lambda m^2$ for some $\lambda \in \ZZ$, leading to 
		\[
			\frac{x_0}{m^2} - \frac{\lambda}{y} = \frac{x}{m^2 y}.
		\]
		The inequalities  $x>0$ and $m^2>2xy$ give the result.
	\end{solution}

	\question Knowing the above, propose a speed-up to the algorithm given in question 2. Argue informally on the achieved gain.
	\begin{solution}
		The modification would be as follows: 
		\begin{enumerate}
			\item Choose $m > 2n^{1/4}$.
			\item Compute all solutions $x_0$ to $Q(x_0, 1) = 0 \bmod m^2$.
			\item For each $x_0$ found, use a continued fraction algorithm to find $\lambda/y < x_0/m^2$ s.t.\ $y < n^{1/4}$. Check if $Q(x_0y - \lambda m^2, y)$ is a square. Try another $m$ if none of $x_0$ worked. 
		\end{enumerate}
	
	\end{solution}
\end{questions}

% \section{How to check if a number is a perfect power}
	
% 	Usually factorization algorithms assume that the input number $N$ we want to factor is not a perfect power. 
% 	Give an efficient algorithm that checks if $N $ is a perfect power, i.e., if $N = a^k$, and outputs the pair $(a,k)$ with the largest $k$.
% 		\begin{solution}
% 			Since $N = a^k$, $k  \le \log N$. For each integer $b$, we try binary search for $a$ from the set $\{ 2, \ldots, \lceil N^{1/b}\rceil\}$. Namely, for $\mathsf{mid} = (\frac{2+N^{1/b}}{2})^b$, check if  $\mathsf{mid}>N$, and if so, continue with the `left' sub-interval), or of $\mathsf{mid}<N$, in which case, continue to the right. Output $(\mathsf{mid}, b)$ in the case of the equality. 
			
% 			We have $< \log N$ possible choices for $b$. For each such choice, the number of $\mathsf{mid}'s$ is $\log(\lceil N^{1/b} \rceil) = \log \log N$. To get a $\mathsf{\mid}$, we raise to the power of $b$ which costs $\log b = \log \log N$ operations (in our case in $\RR$ with some precision). 
			
% 			The algorithm runs in time $O(\log N \cdot (\log \log N)^2)$.
			
% 			Guillaume suggested to take the $b$-th root of N for each choice of $b$. Likely the result won't be an integer (if it is, we're done). We search for a close-by integer. The above is the same where the search for the solution is performed in a binary-search manner.
% 		\end{solution}

\end{document}



%%% Local Variables:
%%% mode: latex
%%% TeX-master: t
%%% End:
