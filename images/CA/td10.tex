\documentclass[11pt]{exam}
%%%%%%%%%%%%%%%%%%%%%%%%%%%%%%%%
\noprintanswers % pour enlever les réponses
%\printanswers

\unframedsolutions
\SolutionEmphasis{\itshape\small}
\renewcommand{\solutiontitle}{\noindent\textbf{A: }}
%%%%%%%%%%%%%%%%%%%%%%%%%%%%%%%%



%\usepackage[margin=0.73in]{geometry}
%\usepackage[top=1in, bottom=1in, left=1in, right=1in]{geometry}

%\usepackage{fullpage}


\usepackage{hyperref}
\usepackage{appendix}
\usepackage{enumerate}


\usepackage{times,graphicx,epsfig,amsmath,latexsym,amssymb,verbatim}%,revsymb}
\usepackage{algorithmicx, enumitem, algpseudocode, algorithm, caption}


%%%%%%%%%%%%%%%%%%%%%
% Handling comments and versions %%%
%%%%%%%%%%%%%%%%%%%%%
\newcommand{\extra}[1]{}

\renewcommand{\comment}[1]{\texttt{[#1]}}

\newcommand*{\ScProd}[2]{\ensuremath{\langle#1\!\mathbin{,}\!#2\rangle}} %Scalar Product


%%%%%%%%%%%%%%%%%%%%%%%%%%%
%% THEOREMS
%%%%%%%%%%%%%%%%%%%%%%%%%%%

\usepackage{amsmath,amssymb,amsfonts}
\usepackage{amsthm}

\newtheorem{theorem}{Theorem}[section]
\newtheorem{axiom}[theorem]{Axiom}
\newtheorem{conclusion}[theorem]{Conclusion}
\newtheorem{condition}[theorem]{Condition}
\newtheorem{conjecture}[theorem]{Conjecture}
\newtheorem{corollary}[theorem]{Corollary}
\newtheorem{criterion}[theorem]{Criterion}
\newtheorem{definition}[theorem]{Definition}
\newtheorem{lemma}[theorem]{Lemma}
\newtheorem{notation}[theorem]{Notation}
\newtheorem{proposition}[theorem]{Proposition}


\theoremstyle{definition}
\newtheorem{problem}{Problem}


\newcommand{\nc}{\newcommand}
\nc{\eps}{\varepsilon}
\nc{\RR}{{{\mathbb R}}}
\nc{\CC}{{{\mathbb C}}}
\nc{\FF}{{{\mathbb F}}}
\nc{\NN}{{{\mathbb N}}}
\nc{\ZZ}{{{\mathbb Z}}}
\nc{\PP}{{{\mathbb P}}}
\nc{\QQ}{{{\mathbb Q}}}
\nc{\UU}{{{\mathbb U}}}
\nc{\OO}{{{\mathbb O}}}
\nc{\EE}{{{\mathbb E}}}


\newcommand{\field}{\mathbb{K}}
\nc{\K}{K}
\newcommand{\M}{\mathcal{M}}
\newcommand{\val}{\operatorname{val}}
\newcommand{\bigO}{\mathcal{O}}
\DeclareMathOperator{\Span}{Span}
\newcommand{\vc}[1]{\mathbf{#1}}

\pretolerance=1000

%%%%%%%%%%%%%%%%%%%%%%%%%%%%%%%%
%%%%%%%%%%%%%%%%%%%%%%%%%%%%%%%%
%% DOCUMENT STARTS
%%%%%%%%%%%%%%%%%%%%%%%%%%%%%%%%
%%%%%%%%%%%%%%%%%%%%%%%%%%%%%%%%
\usepackage{tikz}
\usetikzlibrary{automata}
\DeclareMathOperator{\Vol}{Vol}

\begin{document}

{\noindent
   \textsc{ENS Lyon --  M1 -- Computer Algebra}
   \hfill {E.Kirshanova / A.Wallet // 2017--2018\\
  }
  \hrule
% Titre de la feuille
  \begin{center}
    {\Large\textbf{
   \textsc{Tutorial 10}
    } } 
  \end{center}
  \hrule \vspace{5mm}

  \thispagestyle{empty}

  \section{Time to read your lecture notes}

  \begin{questions}
    \question Let $K$ be a field. Fill the following table \textbf{without} using fast algorithms.

    \begin{tabular}{l|c|c}
      & \hspace{2cm}$\ZZ$\hspace{2cm} & \hspace{2cm}$K[X]$\hspace{2cm} \\[2pt]\hline
      unit operation in complexity &  & \\[2pt]\hline
      input size & & \\[2pt]\hline
      Addition/subtraction & & \\[2pt]\hline
      Multiplication & & \\[2pt]\hline
      Euclidean division & & \\[2pt]\hline
      Extended Euclidean Algorithm & & \\[2pt]\hline
      Multipoints evaluations & & \\[2pt]\hline
      Interpolation/CRT & &
    \end{tabular}

    \question Let {\sc M}$(n)$ be the complexity of multiplying two integers of size $n$, and {\sc P}$(n)$ be the complexity of multiplying two polynomials of degree $n$. What are the best values you know for each?
    \begin{solution}
       {\sc M}$(n) = n\log n\log\log n$ and {\sc P}$(n)=n\log n$.
     \end{solution}

     \question What is the complexity of adding/multiplying polynomials in $\ZZ_p[X]$ in terms of bit operations (and using naive algorithms)?

     \begin{solution}
       The bitsize of a degree $n$ polynomial is $O(n\log p)$. Addition therefore costs $O(n\log p)$ ($n$ additions in $\ZZ_p$), and multiplication of two polynomials of degree $n$ is $O(n^2\log^2 p)$.
     \end{solution}
     

    \question Same as question $1$, but fast algorithms are allowed this time.\\

     \begin{tabular}{l|c|c}
      & \hspace{2cm}$\ZZ$\hspace{2cm} & \hspace{2cm}$K[X]$\hspace{2cm} \\[2pt]\hline
      unit operation in complexity &  & \\[2pt]\hline
      input size & & \\[2pt]\hline
      Addition/subtraction & & \\[2pt]\hline
      Multiplication & & \\[2pt]\hline
      Euclidean division & & \\[2pt]\hline
      Extended Euclidean Algorithm & & \\[2pt]\hline
      Multipoints evaluations & & \\[2pt]\hline
      Interpolation/CRT & &
     \end{tabular}

     \vspace{0.25cm}

     \question Fill the following table for linear algebra.\\
     \begin{tabular}{l|c|c}
      & \hspace{2cm}$\mathcal M_n(K)$\hspace{2cm} & \hspace{1cm}sparse (say, $c$ non-zero coeffs)\hspace{1cm} \\\hline
      unit operation in complexity &  & \\[2pt]\hline
      input size & & \\[2pt]\hline
      Addition/subtraction & & \\[2pt]\hline
      Matrix $\times$ vector & & \\[2pt]\hline
      Matrix multiplication & & \\[2pt]\hline
      Gaussian elimination & & \\[2pt]\hline
      Determinant & & \\[2pt]\hline
       Linear systems & & \\[2pt]\hline
       Inversion & & \\[2pt]\hline
       Characteristic polynomial & & 
     \end{tabular}
     
  \end{questions}
  \section{One-question, one-minute}

  \begin{questions}

    \question Find the smallest $u\geq 0$ such that
    $$\begin{cases} u\equiv 1~[2]\\ u\equiv 2~[3]\\ u\equiv 2~[5]. \end{cases}$$

    \begin{solution}
      $u$ is odd, and finishes by $2$ or $7$. Smallest such number is $17$. 
    \end{solution}

    \question Explain how Karatsuba's algorithm can be seen as an evaluation/interpolation algorithm.

    \begin{solution}
      We can evaluate two polynomial of degree $1$ at three points and interpolate for these three values to obtain the product. A choice can be $0,1,\infty$.
    \end{solution}

    \question Let $K$ be a field, $c=(c_0, \dotsc, c_{n-1}) \in K^n$ and for $a_1, \dotsc, a_n$ elements in $K$, consider the matrix-vector product
    $$\begin{bmatrix} 1 & a_1 & \dots & a_1^{n-1} \\
      1 & a_2 & \dots & a_2^{n-1}\\
      & \vdots & & \\
      1 & a_n & \dots & a_n^{n-1}\end{bmatrix}\cdot \begin{bmatrix} c_0 \\ c_1 \\ \vdots \\ c_n \end{bmatrix}.$$
    What is the best complexity you can achieve? Can you do even better if you ask an additional property on the $a_i$'s?

    \begin{solution}
      This product amounts to evaluate the polynomial $C(X) = \sum_{i=0}^{n-1}c_iX_i$ at the points $a_1, \dotsc, a_n$. Fast multipoint evaluation does it in $O(${\sc M}$(n)\log n)$ operations in $K$. A $\log n$ factor can be removed using FFT if the $a_i$'s are asked to be powers of a root of unity.
      \end{solution}
      
    \end{questions}
  
  \section{Newton's iteration and integer roots}
  Newton's iteration is a fondamental tool in several topics (optimal transport, numerical analysis, ...). It can be seen as an iterative process to approximate a solution of $f(x)=0$, where $f$ is, {\it e.g.~}, a given $\mathcal C^1$- real valued function. Starting from an initial condition $x_0$, one defines
  $$ x_{i+1} = x_i-\frac{f(x_i)}{f'(x_i)}.$$
  Under some reasonable assumptions on $f$, it can be shown that the process converges quadratically fast to a solution. With a bit more work, one can show that Euclidean division of two integers of bitsize $n$ can be done in $O(${\sc M}$(n))$: we will use this result in the exercise.

  We will also show that Newton's iteration can also be used with another notion of convergence to compute integer roots of integer polynomials.
  
  \begin{questions}

    \question Write down Newton's iteration to compute the inverse of a given $a\in \ZZ$. Use it to show that, if you start from an inverse of $a$ modulo some integer $m$, the next iteration gives you an inverse of $a$ modulo $m^2$. Deduce an algorithm to compute inverses modulo $p^v$, for some prime $p$ and an even\footnote{For the sake of simplicity. An analogous algorithm can be designed in the general case (we leave it as an exercise).} integer $v\geq 1$, and give its complexity.
    \begin{solution}
      We want to solve $\frac{1}{x}-a=0$, starting from some initial condition $b_0$. We then have $b_{i+1}=b_i-\frac{1/b_i-a}{-1/b_i^2}=2b_i-b_i^2a$. Now assuming $b_i\cdot a \equiv 1~[m]$, we have $1-b_{i+1}a=1-2b_ia+b_i^2a^2 = (1-b_ia_i)^2 \equiv 0~[m^2]$, which gives the claim.

      We could compute inverse modulo $p^v$ using EEA in $O(\log_2^2(p^v))$ operations. However it is less efficient when $v$ is large than the following algorithm, based on Newton's iteration. In a first step, compute an inverse $b_0$ modulo $p$ with EEA in $O(\log^2 p)$ operations. Then for $i$ from $1$ to $v-1$, let $b_0\gets (2b_0-b_0^2a)\bmod p^{i+1}$, which naively cost $3$ multiplications and one modular reduction: everything can be done in $O(${\sc M}$(\log_2(p^v)))$. This gives $O(\log_2^2p + M(v\log_2 p))$ operations.
    \end{solution}

    \question Let $P \in \ZZ[X]$, and assume that $a_i\in \ZZ$ such that $P(a_i) \equiv 0~[m]$ and $P'(a_i) \in (\ZZ/m\ZZ)^\times$ for some integer $m$. Define the next ``modular'' Newton iterate $a_{i+1}$, and show that it satisfies the following properties:
    \begin{itemize}
    \item $a_{i+1} \equiv a_i~[m]$;
    \item $P(a_{i+1}) \equiv 0~[m^2]$;
    \item $P'(a_{i+1}) \in (\ZZ/m^2\ZZ)^\times$.
    \end{itemize}
    \begin{solution}
      The iteration defines $a_{i+1} = a_i - P(a_i)P'(a_i)^{-1} \bmod m^2$. This directly gives the first property, which also implies that $(a_{i+1}-a_i)^2 \equiv 0~[m^2]$. From Taylor expansion for polynomials, we can write $P(a_{i+1})=P(a_i)+P'(a_i)(a_{i+1}-a_i)+Q(a_i)(a_{i+1}-a_i)^2$, for some polynomial $Q$. By definition of the iterate, we have $-P'(a_i)(a_{i+1}-a_i) \equiv P(a_i)~[m^2]$; we obtain the second claim using the previous remark. Lastly, the fact that $a_{i+1} \equiv a_i~[m]$ and that the reductio modulo $m$ is a ring homomorphism imply that, for any polynomial $Q \in \ZZ[X]$, we have $Q(a_{i+1}) \equiv Q(a_i)~[m]$. In particular, $P'(a_{i+1}) \equiv P'(a_i)~[m]$, so that $P'(a_{i+1})$ is invertible modulo $m$. In other words, it is coprime to $m$, hence coprime to $m^2$, and the last claim follows.
    \end{solution}

    The next question deals with the unicity of the solution given by the above Newton iteration process, assuming $b_0$ is a starting value (thus $P(b_0)\equiv 0~[p]$ and $P'(b_0)$ is invertible modulo $p$).
    \question (\textbf{Uniqueness property})  If $b,b'$ are integers such that $b \equiv b_0 \equiv b'~[p]$ and $P(b) \equiv P(b')~[p^{2^i}]$, show that $b\equiv b'~[p^{2^i}]$.
    \begin{solution}
      Using again Taylor expansion, we may find a polynomial $Q$ such that $P(b')-P(b) = (b-b')(P'(b)-Q(b)(b-b'))$. By assumptions, $P'(b)+Q(b)(b-b') \equiv P'(b) \equiv P'(b_0)~[p]$. In particular, the left hand side is coprime to $p$, hence invertible modulo $p^{2^i}$. Let $u$ be said inverse, so that $u(P(b)-P(b')) \equiv b-b'~[p^{2^i}]$. The claim follows as the left hand side vanishes.
      
    \end{solution}
        

    % \question Design an algorithm that, given an input $a\in \ZZ$ and an odd\footnote{AGain for the sake of simplicity. The even case can be treated as well.} integer $n>2$, certifies that it is not the $n$-th power of an integer, or compute $b\in \ZZ$ such that $b^n =a$.

    % \begin{solution}
    %   The idea is to do ``2-adic'' Newton iteration over $x^n-a$. Take $k=\lceil \log_2\log_2 a\rceil$, and do $2$-adic iterations up to $2^{2^k} > a$: if the last iteration hold exactly, it gives an $n$-th root, else it means $a$ is not an $n$-th power. TODO: write why.

    %   The Newton iteration cannot start if $a$ is even, so we can first compute $v$ (by shifting) such that $a=2^va'$, with $a'$ odd. If $v\neq n$, return ``fail''. Now $a' \equiv 1~[2]$, so we can start our iteration with $b_0=1$. We have $P'(x)=nx^{n-1}$, which is why we asked $n$ to be odd at the first place, so that $u_{0}:= 1 \equiv P'(a)~[2]$. We then compute $b_{i+1}$ and $P'(b_{i})^{-1}$ at each step:
    %   $$ b_{i+1} \gets (b_i-P(b_i)\cdot u_i) \bmod 2^{2^{i+1}},~~u_{i+1} \gets (2u_i-u_i^2P'(b_i)) \bmod 2^{2^{i+1}},$$
    %   where we see from question 1 that $u_{i+1}$ is an inverse of $P'(a)$ modulo $2^{2^{i+1}}$.

    %\end{solution}
    \question What is missing to design a general algorithm to compute ``modular'' roots of integer polynomials? Assuming this problem can be dealt with, write down an algorithm that compute modular roots of integer polynomials and give its complexity.
    
    \begin{solution}
      By CRT, the problem amounts to computing roots modulo prime powers. However, to start Newton's iteration, we require a root modulo $p$. When $p$ is small, a root can be computed by ``brute force'', just as we did above. In general, there are well-known root finding algorithms over field as $\ZZ/p\ZZ$, that were not covered in the lectures this year. Go see Berlekamp's or Cantor-Zassenhaus'.

      If we can find a starting root, we just iterate Newton's until the exponent is reached. At each step, we also update the current inverse for the derivative. This gives the following:
      \begin{itemize}
      \item $a_0 \gets$ Root of $P$ mod $p$;
      \item $u_0 \gets$ ModInverse$(P'(a_0), p)$;
      \item for i from $1$ to $v-1$ do
        \begin{itemize}
        \item $a_i \gets a_{i-1}-P(a_{i-1})u_i\bmod p^{2^i}$, and $u_i\gets 2u_i-u_i^2P'(a_i)\bmod p^{2^i}$
        \end{itemize}
      \item outputs $a_{v-1}$.
      \end{itemize}
      To evaluate the complexity, we use the fact that, by assumption, reduction modulo $p^v$ costs $O(${\sc M}$(v\log p))$. In particular, this means that ``computation mod $p^v$'' can be done in this cost: reduce operands, do the needed operation, reduce the result. In particular, evaluating a polynomial of degree $d$ cost $O(d${\sc M}$(v\log p))$, which turns out to be a bound for any step in the above algorithm. There are $v=\log p$ steps, hence the total complexity is $O(\log p\cdot \deg P${\sc M}$(v\log p))$.

      
    \end{solution}
    
  \end{questions}

  \section{Evaluating the derivatives of a polynomial}

  In this exercise we are give a degree $n$ polynomial $P\in K[X]$, for some field $K$, and $a\in K$. Our goal is to evaluate all the derivative of $P$ at $a$.

  \begin{questions}
    \question Give a naive algorithm to compute $P^{(i)}(a)$ for all $i \in \{0,\dotsc,n\}$. What is its complexity?
    \begin{solution}
      Compute $P^{(i)}$ for all $i$. It take $O(i)$ multiplications to obtain $P^{(i+1)}$ from $P^{(i)}$, making this $O(n^2)$ multiplications in total. Each evalutation can be done in $O(i)$ as well. The total complexity is therefore $O(n^2)$.
    \end{solution}
    
      \question If $a=0$, give an algorithm that computes all $P^{(i)}(0)$'s in linear time.
      \begin{solution}
        We have $P^{(i)}(O)=i!p_i$. From $P^{(i-1)}(0)$, computing this costs 2 multiplications ($i! = (i-1)!\cdot i$ and $P^{(i)}(0)=i!\cdot p_i$.) We compute all needed evaluation in $O(n)$.
      \end{solution}
      

      \question Show that $Q(X)=P(X+a)$ can be computed in quasi-linear time. (Hint: Consider the Euclidean division $P(X)=Q(X)(X-a)^{\deg P/2}+R(X)$, and apply a recursive approach.)

      \begin{solution}
        We have $P(X+a)=Q(X+a)X^{\deg P/2}+R(X+a)$, with $\deg Q, \deg R< n/2$. A divide-and-conquer approach seems in order:
        \begin{itemize}
        \item Compute Euclidan Division of $P$ by $(X-a)^{\deg P/2}$ in $O(${\sc M}$(n))$, and obtain $Q,R$.
        \item Call recursively the algorithm on $Q,R$.
        \item Return $Q(X+a)*X^{\deg P/2}+R(X+a)$.
        \end{itemize}
        The last step can be done in $O(n)$ by shifting-add. The complexity of the algorithm follows $T(n) = 2T(n/2)+O(${\sc M}$(n))$, from wich we obtain $T(n) = O(${\sc M}$(n)\log n)$.
      \end{solution}

      \question Conclude with a quasi-linear time algorithm to compute all $P^{(i)}(a)$'s.

      \begin{solution}
        First compute $Q(X)=P(X+a)$ in quasi-linear time, then evaluate $Q^{(i)}(0)$ for all $i$ in linear time.

      \end{solution}
        

  \end{questions}

    

\vspace{0.2cm}

\end{document}
%%% Local Variables:
%%% mode: latex
%%% TeX-master: t
%%% End:
