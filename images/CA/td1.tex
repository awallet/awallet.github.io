\documentclass[11pt]{exam}
%%%%%%%%%%%%%%%%%%%%%%%%%%%%%%%%
\noprintanswers % pour enlever les réponses
%\printanswers

\unframedsolutions
\SolutionEmphasis{\itshape\small}
\renewcommand{\solutiontitle}{\noindent\textbf{A: }}
%%%%%%%%%%%%%%%%%%%%%%%%%%%%%%%%



\usepackage[margin=0.73in]{geometry}
%\usepackage[top=1in, bottom=1in, left=1in, right=1in]{geometry}

%\usepackage{fullpage}


\usepackage{hyperref}
\usepackage{appendix}
\usepackage{enumerate}


\usepackage{times,graphicx,epsfig,amsmath,latexsym,amssymb,verbatim}%,revsymb}
\usepackage{algorithmicx, enumitem, algpseudocode, algorithm, caption}


%%%%%%%%%%%%%%%%%%%%%
% Handling comments and versions %%%
%%%%%%%%%%%%%%%%%%%%%
\newcommand{\extra}[1]{}

\renewcommand{\comment}[1]{\texttt{[#1]}}


%%%%%%%%%%%%%%%%%%%%%%%%%%%
%% THEOREMS
%%%%%%%%%%%%%%%%%%%%%%%%%%%

\usepackage{amsmath,amssymb,amsfonts}
\usepackage{amsthm}

\newtheorem{theorem}{Theorem}[section]
\newtheorem{axiom}[theorem]{Axiom}
\newtheorem{conclusion}[theorem]{Conclusion}
\newtheorem{condition}[theorem]{Condition}
\newtheorem{conjecture}[theorem]{Conjecture}
\newtheorem{corollary}[theorem]{Corollary}
\newtheorem{criterion}[theorem]{Criterion}
\newtheorem{definition}[theorem]{Definition}
\newtheorem{lemma}[theorem]{Lemma}
\newtheorem{notation}[theorem]{Notation}
\newtheorem{proposition}[theorem]{Proposition}


\theoremstyle{definition}
\newtheorem{problem}{Problem}


\newcommand{\nc}{\newcommand}
\nc{\eps}{\varepsilon}
\nc{\RR}{{{\mathbb R}}}
\nc{\CC}{{{\mathbb C}}}
\nc{\FF}{{{\mathbb F}}}
\nc{\NN}{{{\mathbb N}}}
\nc{\ZZ}{{{\mathbb Z}}}
\nc{\PP}{{{\mathbb P}}}
\nc{\QQ}{{{\mathbb Q}}}
\nc{\UU}{{{\mathbb U}}}
\nc{\OO}{{{\mathbb O}}}
\nc{\EE}{{{\mathbb E}}}

\nc{\K}{K}
\newcommand{\val}{\operatorname{val}}

\pretolerance=1000

%%%%%%%%%%%%%%%%%%%%%%%%%%%%%%%%
%%%%%%%%%%%%%%%%%%%%%%%%%%%%%%%%
%% DOCUMENT STARTS
%%%%%%%%%%%%%%%%%%%%%%%%%%%%%%%%
%%%%%%%%%%%%%%%%%%%%%%%%%%%%%%%%
\usepackage{tikz}
\usetikzlibrary{automata}
\DeclareMathOperator{\Vol}{Vol}

\begin{document}

{\noindent
   \textsc{ENS Lyon --  M1 -- Computer Algebra}
   \hfill { E.Kirshanova / A.Wallet // 2017--2018\\
  }
  \hrule
% Titre de la feuille
  \begin{center}
    {\Large\textbf{
   \textsc{Tutorial 1}
    } } 
  \end{center}
  \hrule \vspace{5mm}

\thispagestyle{empty}

\vspace{0.2cm}

%\Large

\section{Remainder of a sparse polynomial}
{\sl In this exercise we are interested in computing a remainder of a sparse polynomial $S$ after dividing by a polynomial $D$, where $S,D\in K[X]$. (Assume that operations in $K$ have unit cost.)}


\begin{questions}

\question Give an example showing that assuming that $S$ is sparse does not lead to better bounds for the classical division algorithm. 
\begin{solution}$S = X^n$ and $D = X-1$ for a large integer $n$. $S$ is sparse but during naive division, the successive quotients are $X^n$, $X^{n-1}, \dots, X^2, X$.
\end{solution}

\question What is the cost of an operation in $ K[X]/(D(X))$?
\begin{solution}
Multiplication of $P$ and $Q$ in $ K[X]/(D(X))$ is in $O(\deg(D)^2)$ (assuming $P$ and $Q$ are already of degree less than $\deg(D)$). Compute $PQ$ in $K[X]$ in $O(\deg(D)^2)$ (we assumed operations in $K$ are $O(1)$). Then reducing a polynomial of degree less than $2 \deg(D)$ by $D$ costs $O(\deg(D)^2)$. So multiplication in $ K[X]/(D(X))$ can be performed in $O(\deg(D)^2)$. The same is true for addition (the final reduction modulo $D$ gives the $O(\deg(D)^2)$ even if addition in $K[X]$ is linear and not quadratic in the degree).
\end{solution}

\question Show that one can compute $X^N \mod D(X)$ in time $O ((\deg D)^2 \log N)$. (Hint: use fast exponentiation.)
\begin{solution}
Perform fast exponentiation by reducing mod $D$ after each multiplication. We need $\log(N)$ multiplications to compute $X^N$ by fast exponentiation, and each multiplication in $ K[X]/(D(X))$ is in $O(\deg(D)^2)$ (cf previous question) so we get the desired complexity.
\end{solution}

\question Assume that $S$ has $\omega$ nonzero terms. Show that you get an algorithm of complexity $O(\omega (\deg D)^2\log \deg S)$ which beats the classical division for $ \omega$ at most  $\frac{\deg S-\deg D}{\deg D \log \deg S}$.

\begin{solution}
Write $S = \sum_{1 \leq i \leq \omega} s_i X^{n_i}$. For each $i$, compute $X^{n_i}$ modulo $D$ using the algorithm of the previous question. This takes at most $O((\deg D)^2 \log \deg S)$ operations in $K$ for each $X^{n_i}$ so in total it takes $O(\omega (\deg D)^2\log \deg S)$ operations in $K$. Then compute $\sum_{1 \leq i \leq \omega} s_i (X^{n_i} \mod D) \mod D$, which takes $O(\deg(D)^2 \omega)$ operations in $K$ and does not change the previous complexity. So we obtain the desired result.

The classical division algorithm takes $O(\deg D (\deg S-\deg D) )$ operations in $K$. When $\omega \leq \frac{\deg S-\deg D}{\deg D \log \deg S}$, we have $\omega (\deg D)^2\log \deg S \leq \deg D (\deg S -\deg D)$.
\end{solution}
\end{questions}

\section{Exact division}

{\sl The goal of this exercise is to show that we can have a constant
  gain on (naive) polynomial division in the case where we know
  beforehand that the division is exact. As is always the case with
  constant gain, we'll need to argue in the end that our complexity
  model is sound.}
\medskip
Let $A(X) = \sum_{k=0}^{2n-1} a_k X^k$ and $B(X) = \sum_{k=0}^{n-1} b_k X^k$.


\begin{questions}
\question Prove that one can compute $Q_\ell =
\sum_{k=n-\ell}^{n} q_k X^k$ such that $\deg (A - BQ_{\ell}) <
2n-1-\ell$ using $\ell + 1$ divisions and $\ell (\ell + 1)/2$
multiplications in $\K$.
{\sl Note that we are not interested in the remainder $A - BQ_{\ell}$, only in $Q_{\ell}$.}

\begin{solution}
We prove it by induction. For $\ell = 0$, take $Q_l = \frac{a_{2n-1}}{b_{n-1}} X^n$. This needs 1 division and 0 multiplications (and 0 additions).
Assume we have already computed $Q_l$. We know that $A-BQ_l = c X^{2n-2-l} + \dots$. We will take $Q_{l+1} = Q_l + \frac{c}{b_{n-1}}X^{n-1-l}$. So we need to compute $c$. This is one coefficient of $A$ minus one coefficient of $BQ_l$. To compute one coefficient of $BQ_l$, we uses at most $l+1$ multiplications and $l$ additions (because $Q_l$ has at most $l+1$ non zero coefficients). So we need $l+1$ multiplications and $l+1$ additions to compute $c$. So in total, to compute $Q_{l+1}$ we need to compute $Q_l$ plus 1 division, $l+1$ multiplications and $l+1$ additions. When we sum everything, we obtain that we need $l+1$ divisions, $\frac{l(l+1)}{2}$ multiplications and $\frac{l(l+1)}{2}$ additions to compute $Q_l$.
\end{solution}

\question Prove that the algorithm of {\textbf 1.} can be used to compute
$S_\ell = \sum_{k=0}^\ell s_k X^k$ such that $\val(A - S_{\ell}B) > \ell$ using $\ell+1$
divisions and $\ell(\ell + 1)/2$ multiplications in $\K$. (Where $\val(P)$ is the smallest degree of monomial of $P$ with nonzero coefficient.)

\begin{solution}
Take $A' = X^{2n-1} A(\frac{1}{X})$ and $B' = X^{n-1} B(\frac{1}{X})$ (this just reverse the coefficients of $A$ and $B$). Compute $Q_l$ as in question 1 for $A'$ and $B'$ and set $S_l = X^{n} Q_l(1/X)$. Denote by $T = A' - B' Q_l$, which is of degree $\leq 2n-2-l$. We have
\begin{align*}
A - B Q_l &= X^{2n-1} A'(\frac{1}{X}) + X^{n-1} B'(\frac{1}{X}) \cdot X^{n} Q_l(\frac{1}{X}) \\
&= X^{2n-1} T(\frac{1}{X})
\end{align*}

But as $T$ is of degree less that $2n-2-l$, we have that $A-BQ_l$ has a valuation greater than $l+1$ which is what we wanted.
\end{solution}

\question Assume that we know, for some reason, that $B | A$ and
want to compute $A/B$. Use {\textbf 1} \& {\textbf 2} to give an
algorithm for this task, and compare this with the ``schoolbook''
division.

\begin{solution}
As we know that $B$ divides $A$, we also know that if we compute $Q_l$ until $l = n$ we will obtain the quotient $A/B$. So $Q_l$ contains the $l+1$ highest degree coefficients of the quotient $A/B$. In the same way, $S_l$ contains the $l+1$ coefficients of the terms of lower degree. So if we take $l = \lfloor n/2 \rfloor$, we have that $A/B = Q_l \oplus S_l$ (we do not add the central coefficient if it appears in both $Q_l$ and $S_l$, we just take it once).

Complexity: $2(l+1) \sim n$ divisions and $2l(l+1)/2 \sim n^2/4$ multiplications. Compared to the schoolbook division: $n$ divisions and $n^2$ multiplications. We gain a factor 4.
\end{solution}


\question We only counted divisions and multiplications, but in a
standard algebraic model, addition and subtraction also have a
comparable cost. Does our result really make sense?

\begin{solution}
Number of additions in our algorithm: $2l(l+1)/2 \sim n^2/4$ compared to $n^2$ in the schoolbook algorithm. This is again better.
\end{solution}

\end{questions}

%\pagebreak

\section{Multiplication of bivariate polynomials}
Fact: Let $c_0,\ldots c_d$ be $d+1$ distinct elements of $K$ and $Q_0,\ldots Q_d\in K[X]$. There is a unique polynomial $P\in K[X,Y]$ of $Y$-degree at most $d$ satisfying $P(X,c_i)=Q_i$ for every $i=0,\ldots d$.

Let us assume that we can efficiently find such $P$. Again, assume that operations in $K$ have unit cost.


\begin{questions}
\question What is the cost of a naive multiplication of two bivariate polynomials $A$ and $B$ of $X$-degree at most $D_1$ and $Y$-degree at most $D_2$?
\begin{solution}
The monomials of $A$ and $B$ are of the form $X^mY^n$ for $0 \leq m \leq D_1$ and $0 \leq n \leq D_2$. So there are at most $(D_1+1)(D_2+1)$ monomials. And multiplying two polynomials is quadratic in the number of monomials so $O(D_1^2 D_2^2)$.

We can also say that $A,B \in K[X][Y]$ so multiplying them takes $O(D_2^2)$ operations if $K[X]$ and each operation in $K[X]$ is in $O(D_1^2)$ because the degree in $X$ is at most $D_1$.
\end{solution}

\question Give an algorithm that computes $A(X,c)$ for a given $c\in K$, with $A$ of $X$-degree at most $D_1$ and $Y$-degree at most $D_2$. What is its cost?
\begin{solution}
Let $A = \sum_{0 \leq i \leq D_2} A_i(X)Y^i$ with $A_i \in K[X]$ of degree at most $D_1$. To evaluate $A(X,c)$, we let $R = 0$ and for $i = 0$ to $D_2$ do:
\begin{itemize}
\item compute $c^i$, which is $O(1)$ because we already computed $c^{i-1}$ ($i$ is increasing)
\item compute $A_i(X)c^i$ which is $O(D_1)$ operations in $K$
\item add $A_i(X)c^i$ to $R$, which is again $O(D_1)$ operations in $K$.
\end{itemize}
In total, computing $R(X) = A(X,c)$ takes $O(D_1 D_2)$ operations in $K$.
\end{solution}

\question Using the fact above, describe an algorithm for multiplying bivariate polynomials. (Which would, assuming that we have a fast algorithm for multiplication of polynomials of one variable, beat the naive multiplication.)
\begin{solution}
Assume that $|K| \geq 2D_2+1$ and take $c_0, \dots, c_{2D_2}$ distinct elements in $K$. Let $R$ be the ring $K[X]$. This is an integral (and commutative) domain, so if $P \in R[Y]$ is of degree less than $d$ and has $d+1$ roots or more, it is the zero polynomial.

Let $A$ and $B$ seen as polynomials in $R[Y]$. We compute $a_i = A(X,c_i) \in R$ and $b_i = B(X,c_i) \in R$ for all $i$'s. Then, let $Q_i = a_i b_i \in R$ (the $Q_i$'s have degree at most $2D_1$).
Let $P \in K[X,Y]$ be as in the introduction: $P(X,c_i) = Q_i$ for each $i$ and $P$ is of $Y$ degree at most $2D_2$. Then, $P-AB$ seen as a polynomial in $R[Y]$ has degree at most $2D_2$ and is zero on each $c_i$, so this is the zero polynomial, which means that $P = AB$.

Complexity: Computing $A(X,c_i)$ for each $i$ costs $O(D_1 D_2^2)$. The same is true for computing $b_i$. Then, multiplying $a_i$ by $b_i$ is the cost of multiplying two polynomials of degree at most $D_1$ in $K[X]$, which we will call $O(M(D_1))$. This step is done $2D_2$ times, so in total it takes $O(D_2 M(D_1))$ operations in $K$. And we assumed that computing $P$ does not increase the complexity. So in total, we have a complexity $O(D_1 D_2^2 + M(D_1) D_2)$, which beats the naive algorithm (even if $M(D_1) = D_1^2$).
\end{solution}

\end{questions}

\section{Integer division}

Let $\beta = 2^w$ be the machine word; in the sequel, we shall assume
that the processor, given as input $u \in [0, \beta^2 - 1]$ and $v \in
[1, \beta - 1]$, can compute $\lfloor u/v\rfloor$ (the integer part of
$u/v$).

Let $A = \sum_{i=0}^{n} a_i \beta^i$ and $B = \sum_{i=0}^m b_i \beta^i$ be
the expansion of two positive integers in base $\beta$.
We will prove that the following algorithm returns $A/B$ and $A\mod B$.

\begin{algorithm}
\caption*{Integer division}
\begin{algorithmic}[0]
\Procedure{Guess}{$A,B$}
\State $g\gets \lfloor (a_n \beta + a_{n-1}) / b_m \rfloor$
\State \textbf{return} $\min(g,\beta-1)$
\EndProcedure
\medskip

\Procedure{Divide}{$A,B$}
\While{$A \geq B$}
\State $q_{n-m-1}\gets \textsc{Guess}(A,B)$
\State $A \gets A-q_{n-m-1}\beta^{n-m-1}B$
\While{$A<0$} \Comment{Correction loop}
   \State $q_{n-m-1} \gets q_{n-m-1} - 1$
   \State $A \gets A + B \beta^{n-m-1}$
\EndWhile
\EndWhile
\State \textbf{return} $(\sum q_k \beta^k, A)$
\EndProcedure
\end{algorithmic}
\end{algorithm}
%\pagebreak

\begin{questions}
\question Prove that, up to multiplying $A$ and $B$ by suitable
powers of $2$, we can assume that $b_{m} \geq \beta / 2$
\bigskip

\begin{solution}
For some $r \geq 0$, $2^r b_{m} \in [\beta/2, \beta]$.
Then replace $A$ by $2^r A$, $B$ by $2^r B$, and in the result keep
the quotient unchanged and divide the remainder by $2^r$. 
\end{solution}

\question Prove that, up to replacing $A$ by $A - \beta^{n-m}B$, we can
assume that $A < \beta^{n-m} B$.
\bigskip

\begin{solution}
We have $\beta^{n-m} B = b_m \beta^n + \dots$.

If $a_n < b_m$ then we have $A < \beta^{n-m} B$. Otherwise $a_n \geq b_m \geq \beta/2$. Let $A' = A - \beta^{n-m}B$, then $a_n' = a_n - b_m \leq \beta-1-\beta/2 < \beta/2$ and $A' < \beta^{n-m} B$.
If we replace $A$ by $A - \beta^{n-m} B$, we have to add $\beta^{n-m}$ to the quotient and don't change the reminder. 
\end{solution}

Remark: in the previous two questions, also explain how one should
  correct quotient and remainder in the end to get the result of the
  original division...
\bigskip

Define the following loop invariant (for the outer
while loop)~; we denote by $A_k, Q_k$ the value of $A, Q$ after the
$k$-th iteration of the loop (where $Q$ is the integer $\sum_i q_i \beta^i$ for all $q_i$'s that have been defined so far).
\begin{itemize}
\item $0\leq A_k < \beta^{n-m-k} B$
\item $A_k + Q_k B = A$. 
\end{itemize}
where $n$ and $m$ are the maximal powers of $\beta$ that appear in the original $A$ and $B$ (i.e. before the first loop).
\bigskip

\question Deduce that, if the loop invariant is correct, when we exit the
loop $(Q, A)$ are the quotient and the remainder of the Euclidean division
of $A$ by $B$.
\bigskip

\begin{solution}
If the loop invariant is correct, after $n-m$ iterations
we have $A = Q_k B + A_k$ and $0 \leq A_k < B$, qed. 
\end{solution}

\question Prove the second statement of the loop invariant, and the
non-negativity part of the first one.
%\bigskip

\begin{solution}
The second statement is easily checked as each modification
on $A$ induces the corresponding modification on $Q$; the nonnegativity
part is guaranteed by the Correction step. 
\end{solution}
\bigskip

We now turn to proving the loop invariant by induction -- to simplify
the notations we only deal with the first iteration, but the proof
carries over to any iteration. We shall split the proof into two
cases: either $g\leq \beta - 1$ in \textsc{Guess}, or \textsc{Guess} returns $\beta -
1$.
\bigskip

\question First subcase: $g\leq \beta - 1$. Prove that
$q_{n-m-1} b_m \geq a_n \beta + a_{n-1} - b_m + 1$. Deduce that
$$A - q_{n-m-1} B \beta^{n-m-1} \leq \sum_{k=0}^{n-2} a_k \beta^k + (b_m - 1) \beta^{n-1},$$
and the last part of the loop invariant.
\bigskip

\begin{solution}
The first inequality follows from
$q_{n-m-1} = g > (a_n \beta + a_{n-1}) / b_m - 1$, so $q_{n-m-1} b_m > a_n \beta + a_{n-1} - b_m \geq a_n \beta + a_{n-1} - b_m + 1$ because it is an integer.

Then
\begin{eqnarray*}
  A - q_{n-m+1} B \beta^{n-m-1} & \leq & A - q_{n-m+1} b_m \beta^{n-1}\\
  & \leq &
  (A - a_n \beta^n - a_{n-1} \beta^{n-1}) + (b_m - 1) \beta^{n-1}
\end{eqnarray*}
from which the desired identity follows; obviously, the rhs is $< b_m
\beta^{n-1} \leq B \beta^{n-m-1}$, the last part of the loop invariant after
Step 1. 
\end{solution}

\question Second subcase: $g \geq \beta$; prove the last part of the
loop invariant \textit{(hint: use question 2)}. 
%\bigskip

\begin{solution}
In that case, $A - q_{n-m+1} B \beta^{n-m-1} = A - (\beta - 1)
B \beta^{n-m-1} = (A - \beta^{n-m} B) + B \beta^{n-m-1} < B \beta^{n-m-1}$
thanks to question 2.
Note that in Q. 5 and 6, the Correction loop preserves this last part of the loop invariant.
\end{solution}
\bigskip

We now estimate the number of iterations of the Correction loop.
\bigskip

\question Prove that we always have  $\textsc{Guess}(A,B) b_m \leq a_n \beta + a_{n-1},$
and deduce that \[A - \textsc{Guess}(A,B) \beta^{n-m-1} B > -\textsc{Guess}(A,B) \beta^{n-1}.\] Deduce that
the number of iterations of the Correction loop is at most 2 \textit{(hint: use question 1)}.
%\bigskip

\begin{solution}
In all cases, $q_{n-m-1} b_m \leq a_n \beta + a_{n-1}$. From 
$$
A -  q_{n-m-1} \beta^{n-m-1} B > a_n \beta + a_{n-1} - q_{n-m-1} \beta^{n-m-1} ((b_m + 1) \beta^m
$$ we thus deduce that
$A - q_{n-m-1} \beta^{n-m-1} B > -q_{n-m-1} \beta^{n-1} $. As $2b_m \geq \beta$ (using question 1), we deduce that $A - q_{n-m-1} \beta^{n-m-1} B + 2 \beta^{n-m-1} B
\geq (2 b_m - q_{n-m-1}) \beta^{n-m-1} > 0$, so at most 2 correction steps
are required.
\end{solution}
\bigskip

\question Conclude on the correction and complexity of the algorithm.
%\bigskip

\begin{solution}
The correction has been proved in the previous questions; As
for the complexity, Guess has unit cost, the outer loop is performed
$O(n-m+1)$ times, and each operation has cost at most $O(m+1)$ (one needs
to argue a little bit on that point -- no significant outgoing carries /
borrows in the additions that occur).
\end{solution}
\end{questions}


\end{document}


