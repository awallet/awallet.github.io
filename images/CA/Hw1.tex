\documentclass[11pt]{exam}
%%%%%%%%%%%%%%%%%%%%%%%%%%%%%%%%
\noprintanswers % pour enlever les réponses
%\printanswers

\unframedsolutions
\SolutionEmphasis{\itshape\small}
\renewcommand{\solutiontitle}{\noindent\textbf{A: }}
%%%%%%%%%%%%%%%%%%%%%%%%%%%%%%%%



\usepackage[margin=0.99in]{geometry}
%\usepackage[top=1in, bottom=1in, left=1in, right=1in]{geometry}

%\usepackage{fullpage}


\usepackage{hyperref}
\usepackage{appendix}
\usepackage{enumerate}


\usepackage{times,graphicx,epsfig,amsmath,latexsym,amssymb,verbatim}%,revsymb}
\usepackage{algorithmicx, enumitem, algpseudocode, algorithm, caption}


%%%%%%%%%%%%%%%%%%%%%
% Handling comments and versions %%%
%%%%%%%%%%%%%%%%%%%%%
\newcommand{\extra}[1]{}

\renewcommand{\comment}[1]{\texttt{[#1]}}


%%%%%%%%%%%%%%%%%%%%%%%%%%%
%% THEOREMS
%%%%%%%%%%%%%%%%%%%%%%%%%%%

\usepackage{amsmath,amssymb,amsfonts}
\usepackage{amsthm}

\newtheorem{theorem}{Theorem}[section]
\newtheorem{axiom}[theorem]{Axiom}
\newtheorem{conclusion}[theorem]{Conclusion}
\newtheorem{condition}[theorem]{Condition}
\newtheorem{conjecture}[theorem]{Conjecture}
\newtheorem{corollary}[theorem]{Corollary}
\newtheorem{criterion}[theorem]{Criterion}
\newtheorem{definition}[theorem]{Definition}
\newtheorem{lemma}[theorem]{Lemma}
\newtheorem{notation}[theorem]{Notation}
\newtheorem{proposition}[theorem]{Proposition}


\theoremstyle{definition}
\newtheorem{problem}{Problem}


\newcommand{\nc}{\newcommand}
\nc{\eps}{\varepsilon}
\nc{\RR}{{{\mathbb R}}}
\nc{\CC}{{{\mathbb C}}}
\nc{\FF}{{{\mathbb F}}}
\nc{\NN}{{{\mathbb N}}}
\nc{\ZZ}{{{\mathbb Z}}}
\nc{\PP}{{{\mathbb P}}}
\nc{\QQ}{{{\mathbb Q}}}
\nc{\UU}{{{\mathbb U}}}
\nc{\OO}{{{\mathbb O}}}
\nc{\EE}{{{\mathbb E}}}

\nc{\K}{K}
\newcommand{\field}{\mathbb{K}}
\newcommand{\val}{\operatorname{val}}

\pretolerance=1000

%%%%%%%%%%%%%%%%%%%%%%%%%%%%%%%%
%%%%%%%%%%%%%%%%%%%%%%%%%%%%%%%%
%% DOCUMENT STARTS
%%%%%%%%%%%%%%%%%%%%%%%%%%%%%%%%
%%%%%%%%%%%%%%%%%%%%%%%%%%%%%%%%
\usepackage{tikz}
\usetikzlibrary{automata}
\DeclareMathOperator{\Vol}{Vol}

\begin{document}

{\noindent
   \textsc{ENS Lyon --  M1 -- Computer Algebra}
 \hfill { E.Kirshanova / A.Wallet // 2017--2018\\
   	}\\
  }
  \hrule
% Titre de la feuille
  \begin{center}
    {\Large\textbf{
   \textsc{Homework 1} \\[5pt]
    }
	Due:  27.02
	} 
  \end{center}
  \hrule \vspace{5mm}

\thispagestyle{empty}

\vspace{0.2cm}

%\Large

\section{Integer division}

Let $\beta = 2^w$ be the machine word; in the sequel, we shall assume
that the processor, given as input $u \in [0, \beta^2 - 1]$ and $v \in
[1, \beta - 1]$, can compute $\lfloor u/v\rfloor$ (the integer part of
$u/v$).

Let $A = \sum_{i=0}^{n} a_i \beta^i$ and $B = \sum_{i=0}^m b_i \beta^i$ be
the expansion of two positive integers in base $\beta$.
We will prove that the following algorithm returns $A/B$ and $A\mod B$.

{\color{red} In the algorithm \textsc{Integer division} the values $(a_i), n$ are assumed to be consistent with the updates of $A$. I.e., whenever $A \gets ...$, these values are updated.}

\begin{algorithm}
\caption*{Integer division}
\begin{algorithmic}[0]
\Procedure{Guess}{$A,B$}
\State $g\gets \lfloor (a_n \beta + a_{n-1}) / b_m \rfloor$
\State \textbf{return} $\min(g,\beta-1)$
\EndProcedure
\medskip

\Procedure{Divide}{$A,B$}
\While{$A \geq B$}
\State $q_{n-m-1}\gets \textsc{Guess}(A,B)$
\State $A \gets A-q_{n-m-1}\beta^{n-m-1}B$
\While{$A<0$} \Comment{Correction loop}
   \State $q_{n-m-1} \gets q_{n-m-1} - 1$
   \State $A \gets A + B \beta^{n-m-1}$
\EndWhile
\EndWhile
\State \textbf{return} $(\sum q_k \beta^k, A)$
\EndProcedure
\end{algorithmic}
\end{algorithm}
%\pagebreak

\begin{questions}
\question Prove that, up to multiplying $A$ and $B$ by suitable
powers of $2$, we can assume that $b_{m} \geq \beta / 2$
\bigskip

\begin{solution}
For some $r \geq 0$, $2^r b_{m} \in [\beta/2, \beta]$.
Then replace $A$ by $2^r A$, $B$ by $2^r B$, and in the result keep
the quotient unchanged and divide the remainder by $2^r$. 
\end{solution}

\question Prove that, up to replacing $A$ by $A - \beta^{n-m}B$, we can
assume that $A < \beta^{n-m} B$.
\bigskip

\begin{solution}
We have $\beta^{n-m} B = b_m \beta^n + \dots$.

If $a_n < b_m$ then we have $A < \beta^{n-m} B$. Otherwise $a_n \geq b_m \geq \beta/2$. Let $A' = A - \beta^{n-m}B$, then $a_n' = a_n - b_m \leq \beta-1-\beta/2 < \beta/2$ and $A' < \beta^{n-m} B$.
If we replace $A$ by $A - \beta^{n-m} B$, we have to add $\beta^{n-m}$ to the quotient and don't change the reminder. 
\end{solution}

Remark: in the previous two questions, also explain how one should
  correct quotient and remainder in the end to get the result of the
  original division...
\bigskip

Define the following loop invariant (for the outer
while loop)~; we denote by $A_k, Q_k$ the value of $A, Q$ after the
$k$-th iteration of the loop (where $Q$ is the integer $\sum_i q_i \beta^i$ for all $q_i$'s that have been defined so far).
\begin{itemize}
\item $0\leq A_k < \beta^{n-m-k} B$
\item $A_k + Q_k B = A$. 
\end{itemize}
where $n$ and $m$ are the maximal powers of $\beta$ that appear in the original $A$ and $B$ (i.e. before the first loop).
\bigskip

\question Deduce that, if the loop invariant is correct, when we exit the
loop $(Q, A)$ are the quotient and the remainder of the Euclidean division
of $A$ by $B$.
\bigskip

\begin{solution}
If the loop invariant is correct, after $n-m$ iterations
we have $A = Q_k B + A_k$ and $0 \leq A_k < B$, qed. 
\end{solution}

\question Prove the second statement of the loop invariant, and the
non-negativity part of the first one.
%\bigskip

\begin{solution}
The second statement is easily checked as each modification
on $A$ induces the corresponding modification on $Q$; the nonnegativity
part is guaranteed by the Correction step. 
\end{solution}
\bigskip

We now turn to proving the loop invariant by induction -- to simplify
the notations we only deal with the first iteration, but the proof
carries over to any iteration. We shall split the proof into two
cases: either $g\leq \beta - 1$ in \textsc{Guess}, or \textsc{Guess} returns $\beta -
1$.
\bigskip

\question First subcase: $g\leq \beta - 1$. Prove that
$q_{n-m-1} b_m \geq a_n \beta + a_{n-1} - b_m + 1$. Deduce that
$$A - q_{n-m-1} B \beta^{n-m-1} \leq \sum_{k=0}^{n-2} a_k \beta^k + (b_m - 1) \beta^{n-1},$$
and the last part of the loop invariant.
\bigskip

\begin{solution}
The first inequality follows from
$q_{n-m-1} = g > (a_n \beta + a_{n-1}) / b_m - 1$, so $q_{n-m-1} b_m > a_n \beta + a_{n-1} - b_m \geq a_n \beta + a_{n-1} - b_m + 1$ because it is an integer.

Then
\begin{eqnarray*}
  A - q_{n-m+1} B \beta^{n-m-1} & \leq & A - q_{n-m+1} b_m \beta^{n-1}\\
  & \leq &
  (A - a_n \beta^n - a_{n-1} \beta^{n-1}) + (b_m - 1) \beta^{n-1}
\end{eqnarray*}
from which the desired identity follows; obviously, the rhs is $< b_m
\beta^{n-1} \leq B \beta^{n-m-1}$, the last part of the loop invariant after
Step 1. 
\end{solution}

\question Second subcase: $g \geq \beta$; prove the last part of the
loop invariant \textit{(hint: use question 2)}. 
%\bigskip

\begin{solution}
In that case, $A - q_{n-m+1} B \beta^{n-m-1} = A - (\beta - 1)
B \beta^{n-m-1} = (A - \beta^{n-m} B) + B \beta^{n-m-1} < B \beta^{n-m-1}$
thanks to question 2.
Note that in Q. 5 and 6, the Correction loop preserves this last part of the loop invariant.
\end{solution}
\bigskip

We now estimate the number of iterations of the Correction loop.
\bigskip

\question Prove that we always have  $\textsc{Guess}(A,B) b_m \leq a_n \beta + a_{n-1},$
and deduce that \[A - \textsc{Guess}(A,B) \beta^{n-m-1} B > -\textsc{Guess}(A,B) \beta^{n-1}.\] Deduce that
the number of iterations of the Correction loop is at most 2 \textit{(hint: use question 1)}.
%\bigskip

\begin{solution}
In all cases, $q_{n-m-1} b_m \leq a_n \beta + a_{n-1}$ (by definition of $\textsc{Guess}$), from where is follows that $q_{n-m-1} b_m \cdot \beta^{n-1} \leq a_n \beta^n + a_{n-1}\beta^{n-1}$. Since $A \geq a_n \beta^{n} + a_n \beta^{n-1}$ and $B < (b_m+1) \beta^m$, we have 
$$
A -  q_{n-m-1} \beta^{n-m-1} B > a_n \beta + a_{n-1} - q_{n-m-1} \beta^{n-m-1} (b_m + 1) \beta^m
$$ we thus deduce that
$A - q_{n-m-1} \beta^{n-m-1} B > -q_{n-m-1} \beta^{n-1} $. As $2b_m \geq \beta$ (using question 1), we deduce that $A - q_{n-m-1} \beta^{n-m-1} B + 2 \beta^{n-1} B
\geq (2 b_m - q_{n-m-1}) \beta^{n-m-1} > 0$, so at most 2 correction steps
are required.
\end{solution}
\bigskip

\question Conclude on the correction and complexity of the algorithm.
%\bigskip



\begin{solution}
The correction has been proved in the previous questions. As
for the complexity, Guess has unit cost, the outer loop is performed
$O(n-m+1)$ times, and each operation has cost at most $O(m+1)$ (one needs
to argue a little bit on that point -- no significant outgoing carries /
borrows in the additions that occur).
\end{solution}
\end{questions}

\section{Multiplication of two polynomials}

Give an algorithm to multiply a degree $1$ polynomial by a degree $2$
polynomial in at most $4$ multiplications. %5[2013 midterm exam, 20% correct
%answers].

\begin{solution}
	Let $P = a + bX$ and $Q = c + d X + e X^2$. We denote by $R = PQ$, the product of $P$ and $Q$ of degree 3. To determine $R$, we need to evaluate it on 4 points. We choose $0,1,-1$ and $+\infty$.
	We obtain
	$$ R = ac + (ad+bc)X + (ae+bd) X^2 + be X^3$$
	and we have
	\begin{align*}
	R(0) &= P(0) Q(0) = ac \\
	R(1) &= P(1) Q(1) = (a+b)(c+d+e) = ac + ad + ae + bc + bd + be \\
	R(-1) &= P(-1) Q(-1) = (a-b)(c-b+e) = ac - ad + ae - bc + bd - be \\
	R(+\infty) &= P(+\infty) Q(+\infty) = be
	\end{align*}
	
	So to conclude, we reconstruct $R$ by computing 
	$$ R = R(0) + \frac{R(1)-R(-1)-2R(+\infty)}{2} X + \frac{R(1)+R(-1)-2R(0)}{2} X^2 + R(+\infty) X^3$$
	Total: 4 multiplications in $K$ and a bunch of additions/subtractions and divisions by 2 (which we assume is negligible compared to multiplications).
\end{solution}

\section{Composition of polynomials}

\paragraph{A remark for the future:} this exercise is not particularly well formulated. Many are getting confused and do not come up with the right algorithm: students are hinted at splitting $g$ and try to divide-and-conquer on $g$, not on $f$. They do not realize that $m$ remains fixed until the end of the analysis and is optimized for only after $C(n,N)$ is deduced.

\begin{questions}
	\question What is the cost of computing the coefficients of the composition $f\circ g$ of  polynomials $f,g$ of degrees $d_1, d_2$? (Assume that ring operations have unit cost.)
	Use that $f(x)=\sum_{i=0}^{d_1}a_ix^i=a_0+x(a_1+x(a_2+\cdots+x(a_{d_1-1}+a_{d_1} x)\ldots)$.


	\begin{solution}
		A naive algorithm if of complexity $\sum_{i=1}^{d_1-1}(M(d_2, id_1) \leq d_1 M(d_1 d_2)$.
		A less naive method is described in Knuth (vol2, Chapter 4.7). For $d_1=d_2$ this method computes the $n$-th coefficient using $2n$ multiplications in $\K$ (i.e, $O(n^3)$ for naive multiplication, and $O(n^2 \log n)$ when we use FFT).
	\end{solution}
	
	Let $N>0$ be a power of 2 and let
	$A$ and $B$ be two polynomials  
	over $\field$ with $B(0) = 0$ and $B'(0) \neq 0$.
	We will study a fast algorithm
	for computing the composition
	$A(B) \mod X^N$ which is due
	to Brent and Kung (1978).
	
	Let $m>0$ be a parameter which we will
	tune later. The algorithm is based on the following
	Taylor's expansion.
	
	\question
	Writing $B = B_1 + X^m B_2$ where $B_1$ is a polynomial
	of degree $<m$ in $\field[X]$, show that 
	\[A(B) \;\;=\;\; A(B_1) \;+\; A'(B_1) X^m B_2 \;+\; A''(B_1) \frac{X^{2m} B_2^2}{2} 
	\;+\; A^{(3)}(B_1) \frac{X^{3m} B_2^3}{6} \;+\; \cdots\]
	
	\begin{solution}
		The paper \url{http://www.eecs.harvard.edu/~htk/publication/1978-jacm-brent-kung.pdf} describes the solution.
		There are several possible way to get the result. One is simpler than the other, I sketch only some of them. First, one looks at Tailor series of $A(X)$ (it is enough to consider the sum only up to $\deg A$ since $A^{(\deg A) = 0}$): 
		\[
			A(X_0+X) = \sum_{i=0}^{\deg A} A^{(i)}(X_0) \frac{X^i}{i!}. 
		\]
		One sets $X_0 = B_1(z), X = X^m B_2(z)$ to obtain the result.
		
		Another (longer) way to obtain the result is by induction on the degree of $A$. Let $n = \deg A$. For $n=0$, the result holds trivially. Otherwise, we have $A = \sum_{i=0}^{n+1} a_i X^i$ and set $C(X) = A(X) - a_{n+1}X^{n+1}, D(X) = a_{n+1}X^{n+1}.$ Expanding $A(B) = C(B)+D(B)$ using the hypothesis on $C(X)$ and re-arranging the terms gives the result.
		
	\end{solution}
	Having this decomposition, let us now observe that once we have computed the 
	composition $A(B_1)$ we can compute the other terms efficiently.
	
	\question
	Let $F$ and $G$ be two polynomials with $G'(0) \neq 0$, and assume 
	that we have computed $F(G) \mod X^N$.
	Show how to compute $F'(G) \mod X^N$ using $\mathcal{O}(\mathrm{M}(N))$
	operations in $\field$, where $\mathrm{M}(n)$ stands for the complexity of multiplying 
	two polynomials of degree $n$ over $\field$.
	\begin{solution}
		Use Lemmas 2.2 and 2.3 from the link above.
		In more details, use the chain rule to get $\frac{dF(G)}{dX} = G' \cdot F'(G)$.Hence, $F'(G)$ can be computed as a quotient of two polynomials.
	\end{solution}
	\question
	Denoting by $\mathcal{C}(m,N)$ the number of operations
	used for computing $A(B_1)\mod X^N$, deduce from the
	previous question a cost bound for computing $A(B) \mod X^N$.
	
	
	To obtain a fast algorithm, it remains to give an efficient
	method to compute $A(B_1) \mod X^N$. To this end, we will
	study the following more general situation.
		\begin{solution}
			Using the previous question, we compute $A^{(i+1)}(B_1) \bmod X^N$ using $A^{(k)}(B_1) \bmod X^N$ in $O(M(N))$. Then we compute $\frac{X^{(k+1)m} B_2^{k+1}}{(k+1)!} \bmod X^N$ from $\frac{X^{km}B_2^k}{k!}$ in time $O(M(N))$. Finally, adding the obtained terms is fast. 
			
			Important to notice that for $i > \lceil N/m \rceil$, we have $\deg (X^i) >N$, so it makes sense to consider only $i \leq \lceil N/m \rceil$. In which case, the total cost of the algorithm is $O \left( C(m,N)+\frac{N}{m}M(N) \right)$. The next question determines $C(m,N)$.
		\end{solution}
	\question
	Let $F$ and $G$ be polynomials over $\field$ of degrees $k$ 
	and $m$ respectively, with $G(0) = 0$.
	Give a divide-and-conquer algorithm which computes $F(G) \mod X^N$
	using $\mathcal{O}(\frac{km}{N} \mathrm{M}(N)\log(N))$ operations in $\field$.
	\begin{solution}
		We split $F(G) = F_1(G)+G^{k/2}F_2(G)$ with $\deg_F1, \deg F_2 \leq k/2$. If $T(k) $ denotes the time to compute $G^{k/2}$ and $F(G)$, we get the recurrence $T(k) \leq 2 T(k/2)+O(M(\min(km, N)))$.
	
	\end{solution}
	\question
	Deduce an upper bound for $\mathcal{C}(m,N)$,
	and a cost bound for computing $A(B) \mod X^N$.
	Conclude by giving the whole algorithm,
	including a good choice of $m$ and the corresponding
	cost bound.
\end{questions}


\end{document}



%%% Local Variables:
%%% mode: latex
%%% TeX-master: t
%%% End:
