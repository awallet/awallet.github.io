%% start of file `template.tex'.
%% Copyright 2006-2015 Xavier Danaux (xdanaux@gmail.com).
%
% This work may be distributed and/or modified under the
% conditions of the LaTeX Project Public License version 1.3c,
% available at http://www.latex-project.org/lppl/.


\documentclass[10pt,a4paper, roman]{moderncv}        % possible options include font size ('10pt', '11pt' and '12pt'), paper size ('a4paper', 'letterpaper', 'a5paper', 'legalpaper', 'executivepaper' and 'landscape') and font family ('sans' and 'roman')

% moderncv themes
\moderncvstyle{classic}                             % style options are 'casual' (default), 'classic', 'banking', 'oldstyle' and 'fancy'
\moderncvcolor{burgundy}                               % color options 'black', 'blue' (default), 'burgundy', 'green', 'grey', 'orange', 'purple' and 'red'
%\renewcommand{\familydefault}{\sfdefault}         % to set the default font; use '\sfdefault' for the default sans serif font, '\rmdefault' for the default roman one, or any tex font name
%\nopagenumbers{}                                  % uncomment to suppress automatic page numbering for CVs longer than one page

% character encoding
\usepackage[utf8]{inputenc}                       % if you are not using xelatex ou lualatex, replace by the encoding you are using
%\usepackage{CJKutf8}                              % if you need to use CJK to typeset your resume in Chinese, Japanese or Korean

% adjust the page margins
\usepackage[scale=0.81]{geometry}
\setlength{\hintscolumnwidth}{3cm}                % if you want to change the width of the column with the dates
%\setlength{\makecvtitlenamewidth}{10cm}           % for the 'classic' style, if you want to force the width allocated to your name and avoid line breaks. be careful though, the length is normally calculated to avoid any overlap with your personal info; use this at your own typographical risks...

% personal data
\name{Alexandre}{Wallet}
\title{Ph.~D.~in computer science\\"Agrégé'' in mathematics}                               % optional, remove / comment the line if not wanted

%\address{90, Rue Racine}{69100 Villeurbanne}{}% optional, remove / comment the line if not wanted; the "postcode city" and "country" arguments can be omitted or provided empty
\phone[mobile]{(+33) 637572324}                   % optional, remove / comment the line if not wanted; the optional "type" of the phone can be "mobile" (default), "fixed" or "fax"
%\phone[fixed]{+2~(345)~678~901}
%\phone[fax]{+3~(456)~789~012}

\email{wallet.alexandre@gmail.com}                               % optional, remove / comment the line if not wanted
\homepage{http://awallet.github.io}                         % optional, remove / comment the line if not wanted
%\social[linkedin]{alexandre.wallet}                        % optional, remove / comment the line if not wanted
%\social[twitter]{jdoe}                             % optional, remove / comment the line if not wanted
%\social[github]{jdoe}                              % optional, remove / comment the line if not wanted
%\extrainfo{}                 % optional, remove / comment the line if not wanted
%\photo[64pt][0.4pt]{picture}                       % optional, remove / comment the line if not wanted; '64pt' is the height the picture must be resized to, 0.4pt is the thickness of the frame around it (put it to 0pt for no frame) and 'picture' is the name of the picture file
%\quote{Some quote}                                 % optional, remove / comment the line if not wanted

% bibliography adjustements (only useful if you make citations in your resume, or print a list of publications using BibTeX)
%   to show numerical labels in the bibliography (default is to show no labels)
%\makeatletter\renewcommand*{\bibliographyitemlabel}{\@biblabel{\arabic{enumiv}}}\makeatother
%   to redefine the bibliography heading string ("Publications")
%\renewcommand{\refname}{Articles}

% bibliography with mutiple entries
%\usepackage{multibib}
%\newcites{book,misc}{{Books},{Others}}
%----------------------------------------------------------------------------------
%            content
%----------------------------------------------------------------------------------
\begin{document}
%\begin{CJK*}{UTF8}{gbsn}                          % to typeset your resume in Chinese using CJK
%-----       resume       ---------------------------------------------------------
\makecvtitle

\section{Current position}
\cventry{}{Post-doctoral researcher}{}{\'Ecole Normale Supérieure de Lyon, LIP, team AriC}{}{Post-quantum cryptology, lattices, algebraic number theory}

\section{Scientific interests}
\cvlistdoubleitem{Algebraic geometry}{Number theory}
\cvlistdoubleitem{Cryptology}{Computer security}
\cvlistdoubleitem{Computer algebra}{Algorithmic}

\section{Education}
\cventry{2013\,--\,2016}{Ph.~D. in computer science}{}{Université Pierre et Marie Curie, Sorbonne, Paris}{}{Thesis: {\em ``Le problème de décomposition de points dans les variétés Jacobiennes''}\\Advisor: J-C.~Faugère, Supervisor: V.~Vitse} 
\cventry{September 2012}{Master degree in fundamental mathematics}{}{\'{E}cole Normale Supérieure de Lyon}{}{Memoir: {\em ``\'Eléments de K-théorie des C*-algèbres''}. }
\cventry{July 2011}{``Agrégation'' in mathematics}{}{prepared at Université Claude Bernard, Lyon 1}{}{Highly selective nation-wide qualification on mathematics at post-graduate level}%Ranked: 128/288}  
\cventry{September 2010}{Master degree in applied mathematics}{}{Université Claude Bernard, Lyon 1}{}{Memoir: {\em ``Introduction au problème du logarithme discret''}. }

% \section{Thèse de doctorat}
% \cventry{Titre}{Le problème de décomposition de points dans les variétés Jacobiennes}{}{}{}{}
% \cventry{Directeur}{\normalfont Jean-Charles Faugère}{}{}{}{}
% \cventry{Encadrante}{\normalfont Vanessa Vitse}{}{}{}{}
% \cventry{Résumé}{\normalfont Le cadre général est l'algorithmique du calcul de logarithmes discrets dans les variétés abéliennes. Des candidates naturelles sont les variétés Jacobiennes des courbes algébriques, particulièrement les courbes elliptiques. J'ai généralisé et formalisé certains algorithmes prometteurs pour les courbes hyperelliptiques, et aussi amélioré des algorithmes connus dans divers contextes. Certaines améliorations permettent d'envisager des calculs se comparant aux records actuels du domaine, et ont été accompagnées d'implémentations dans des langages de calcul formel}{}{}{}{}

\medskip
\section{Supervision of students}
\cventry{April 2018, \\4 months}{\normalfont Thanh Huyen Nguyen, research internship at \'{E}cole Normale Sup\'{e}rieure de Lyon}{}{}{}{In collaboration with E.~Kirshanova and D.~Stehlé}

\medskip
\section{Journal articles}
\cventry{Published}{\normalfont The Point Decomposition Problem in the divisor class group of hyperelliptic curves: toward efficient computations in even characteristic}{with J-C.~Faugère, Design, Codes and Cryptography (DCC)}{}{}{}

\medskip
\section{Peer-reviewed conferences}
\cventry{Published}{\normalfont On the Ring-LWE and Polynomial-LWE problems}{with M. Rosca and D. Stehlé, International Conference on Cryptology and Information Security, EUROCRYPT 2018}{}{}{}
\cventry{Published}{\normalfont Improved Sieving on Algebraic Curves}{with V. Vitse, International Conference on Cryptology and Information Security in Latin America, LATINCRYPT 2015}{}{}{}

%\cvitem{supervisors}{Supervisors}
%\cvitem{description}{Short thesis abstract}
% Vitse, Vanessa, and Alexandre Wallet. "Improved Sieving on Algebraic Curves." . Springer International Publishing, 2015.

\newpage
\section{Selected presentations}

\paragraph{\textbf{Algebraic aspects of ``Learning with errors''}}\medskip

\cventry{11 September 2018}{\normalfont Cryptology and security seminar NTT, Tokyo, Japan}{}{}{}{}
\cventry{15 June 2018}{\normalfont CCA Seminar, INRIA Center, Paris, France}{}{}{}{}
\cventry{20 October 2017}{\normalfont Lattice Meetings, ENS Lyon, LIP, France}{}{}{}{}
\medskip

\paragraph{\textbf{Discrete logarithm over algebraic curves}}\medskip

\cventry{17 May 2017}{\normalfont ECO/ESCAPE Seminar, LIRMM, Montpellier, France}{}{}{}{}
\cventry{24 April 2017}{\normalfont National days of Coding et Cryptograpy, La Bresse, France}{}{}{}{}
\cventry{14 March 2017}{\normalfont National days of the Mathematical Computer Science society, LIRMM, Montpellier, France}{}{}{}{}
%\cventry{8 December 2016}{\normalfont ``Décompositions de points dans les variétés Jacobiennes''}{Séminaire GTBAC, Télécom ParisTech, Paris}{}{}{}
\cventry{25 August 2015}{\normalfont LATINCRYPT 2015}{Guadalajara, Mexico}{}{}{}{}
%\cventry{12 October 2014}{\normalfont ``Finding relations in Index-Calculus for hyperelliptic Jacobian varieties ''}{Séminaire POLSYS, LIP6, Paris}{}{}{}

\medskip
\section{Professional and scientific experiences}
\cventry{2012 -- 2013}{Maths teacher}{Parc Chabrières Highschool}{Oullins, France}{}{}
\cventry{May 2012,\\ 4 months}{Research internship}{Camille Jordan Institute}{Lyon, France}{}{Topic: K-theory for $C^*$-algebras and non-commutative index theory. Supervisor: D.~Perrot}
\cventry{May 2010,\\ 4 months}{Research internship}{Camille Jordan Institute}{Lyon, France}{}{Topic: Introduction to the discrete logarithm problem. Supervisor: C.~Delaunay}

\medskip
\section{Teachings}
\cventry{2018\\2nd semester}{Teacher assistant in Computer Science}{}{\'{E}cole Normale Supérieure de Lyon, 69}{}{
  \begin{itemize}
  \item Tutorials in Computer Algebra in master degree
  \item Evaluation of undergraduate interns
  \end{itemize}
}
\cventry{2013\,--\,2016}{Teacher assistant in bachelor of computer science}{}{Université Pierre et Marie Curie, Sorbonne, Paris}{}{
  \begin{itemize}
  \item 3rd year: Introduction to Cryptology  
  \item 2nd year: Scientific computations , Types and Data structures in C,\\ Machine Architecture and Representation , Development and compilation environment , Discrete structures 
  \item 1st year: Introduction to programming with Python 
  \end{itemize}
}
\cventry{Other}{\normalfont Master SFPN of Université Pierre et Marie Curie, LIP6, specialization in Computer security and Cryptology}{}{}{}{
  \begin{itemize}
  \item Elaboration of exams
  \item Realization of a Side-Channel Attack (SCA) on a faulty AES implementation
  \end{itemize}
}
\cventry{2012 -- 2013}{Maths Teacher}{Parc Chabrières Highschool}{Oullins, 69}{}{
    \begin{itemize}
    \item Full responsibility of two classes for an entire year: lectures and exercises, homeworks, exams.
    \item Trimestrial meetings with the team of teachers and the hierarchy.
    \item Relationships with parents, orientation of students.
    \end{itemize}
  }

\medskip
\section{Skills}
%\cvitem{Research related}{Abelian varieties, algebraic curves, Algebraic and Algorithmic Number Theory, Polynomial Systems, Cryptology, Computer Algebra.}  
\cvitem{Programming}{Basic skills in C, C++, Assembler (8051, x86, MIPS), Python, Shell}
\cvitem{Computer algebra}{Magma, Maple, Sage}
\cvitem{Environments}{Windows, Linux}
\cvitem{Other}{Basic skills in reverse-engineering, web-security fault exploitations and injections.}


\section{Languages}
\cvlistdoubleitem{French: native}{Japanese: school level (B1)}
\cvlistdoubleitem{English: full professionnal proficiency}{Russian: school level (A2)}
\cvlistitem{German: school level (B1)}%{Russe: Notions}

% \section{Interêts et Loisirs}
% \cvitem{Sport}{Badminton (Lycée sport-étude), Boxe Thaïe}
% \cvitem{Musique}{Electronique, Métal}


% \cvitem{hobby 2}{Description}
%\cvitem{hobby 3}{Description}

% \section{Extra 1}
% \cvlistitem{Item 1}
% \cvlistitem{Item 2}
% \cvlistitem{Item 3. This item is particularly long and therefore normally spans over several lines. Did you notice the indentation when the line wraps?}

% \section{Extra 2}
% \cvlistdoubleitem{Item 1}{Item 4}
% \cvlistdoubleitem{Item 2}{Item 5\cite{book1}}
% \cvlistdoubleitem{Item 3}{Item 6. Like item 3 in the single column list before, this item is particularly long to wrap over several lines.}

% \section{References}
% \begin{cvcolumns}
%   \cvcolumn{Category 1}{\begin{itemize}\item Person 1\item Person 2\item Person 3\end{itemize}}
%   \cvcolumn{Category 2}{Amongst others:\begin{itemize}\item Person 1, and\item Person 2\end{itemize}(more upon request)}
%   \cvcolumn[0.5]{All the rest \& some more}{\textit{That} person, and \textbf{those} also (all available upon request).}
% \end{cvcolumns}

% % Publications from a BibTeX file without multibib
% %  for numerical labels: \renewcommand{\bibliographyitemlabel}{\@biblabel{\arabic{enumiv}}}% CONSIDER MERGING WITH PREAMBLE PART
% %  to redefine the heading string ("Publications"): \renewcommand{\refname}{Articles}
% \nocite{*}
% \bibliographystyle{plain}
% \bibliography{publications}                        % 'publications' is the name of a BibTeX file

% Publications from a BibTeX file using the multibib package
%\section{Publications}
%\nocitebook{book1,book2}
%\bibliographystylebook{plain}
%\bibliographybook{publications}                   % 'publications' is the name of a BibTeX file
%\nocitemisc{misc1,misc2,misc3}
%\bibliographystylemisc{plain}
%\bibliographymisc{publications}                   % 'publications' is the name of a BibTeX file

\clearpage
%-----       letter       ---------------------------------------------------------
%recipient data
% \recipient{Company Recruitment team}{Company, Inc.\\123 somestreet\\some city}
% \date{January 01, 1984}
% \opening{Dear Sir or Madam,}
% \closing{Yours faithfully,}
% \enclosure[Attached]{curriculum vit\ae{}}          % use an optional argument to use a string other than "Enclosure", or redefine \enclname
% \makelettertitle

% \makeletterclosing

%\clearpage\end{CJK*}                              % if you are typesetting your resume in Chinese using CJK; the \clearpage is required for fancyhdr to work correctly with CJK, though it kills the page numbering by making \lastpage undefined
\end{document}



%%% Local Variables:
%%% mode: latex
%%% TeX-master: t
%%% End:
