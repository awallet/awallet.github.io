% \documentclass[11pt]{article}
% \usepackage[utf8]{inputenc}
% \usepackage[francais]{babel}
% \usepackage{mathptmx}
% \usepackage[T1]{fontenc}
% \usepackage{amsmath,amsfonts}
% \usepackage{pdfpages}
% \usepackage[scale=0.8]{geometry}
% \usepackage{hyperref}
% \usepackage{enumitem}
% \usepackage{multirow}
% \usepackage{multicol}
% \usepackage{booktabs}
% \usepackage{tikz}
% \usepackage{combelow}

\documentclass[11pt]{article}
\usepackage[dvipsnames]{xcolor}
\usepackage[utf8]{inputenc}
\usepackage[french]{babel}
\usepackage{mathptmx}
\usepackage{libertine}
%\usepackage{crimson}
\usepackage[T1]{fontenc}
\usepackage{amsmath,amsfonts}
\usepackage{pdfpages}
\usepackage[scale=0.8]{geometry}
\usepackage[colorlinks=true,urlcolor=RoyalBlue,citecolor=OrangeRed,linkcolor=Emerald]{hyperref}
\usepackage{enumitem}
\usepackage{multirow}
\usepackage{multicol}
\usepackage{booktabs}
\usepackage{tikz}
\usepackage{combelow}
\usepackage{colortbl}
\usepackage{listings}


\tikzstyle{mat}=[rectangle,rounded corners=8 ,draw=black!100,fill=white!20,thick, inner sep=0pt,minimum height=1cm,minimum width=2.3cm]
\tikzstyle{big}=[rectangle,rounded corners=8,draw=black!100,fill=white!20,thick,dashed, inner sep=0pt,minimum height=7.3cm,minimum width=7.9cm]
\tikzstyle{arr}=[rectangle,draw=white!100,fill=white!20,thick, inner sep=0pt,minimum height=1cm,minimum width=0.6cm]

\newcommand{\SVP}{\ensuremath{\mathsf{SVP}}}
\newcommand{\MSVP}{\ensuremath{\text{Module-}\mathsf{SVP}}}
\newcommand{\ISVP}{\ensuremath{\text{Ideal-}\mathsf{SVP}}}
\newcommand{\PLWE}{\ensuremath{\text{Poly-}\mathsf{LWE}}}
\newcommand{\RLWE}{\ensuremath{\text{Ring-}\mathsf{LWE}}}
\newcommand{\MLWE}{\ensuremath{\text{Module-}\mathsf{LWE}}}
\newcommand{\LWE}{\ensuremath{\mathsf{LWE}}}
\newcommand{\cs}{\cb{s}}
\newcommand{\falcon}{\textsc{Falcon}}
\title{Curriculum Vitae}
\author{Alexandre Wallet}
\date{}



\begin{document}
\maketitle
\tableofcontents

\section{Renseignements généraux}
%Nom, prénom: Wallet, Alexandre\\
\begin{multicols}{2}
\noindent Date de naissance: 3 Février 1986\\
Nationalité: Française\\[10pt]
Qualifications MCF: sections 25, 26, 27

% Adresse professionnelle:\\
% NTT Okamoto Research Laboratory\\
% Secure Platform Laboratories\\
% 3-9-11, Midori-cho, Musashino-shi\\
% Tokyo 180-0012, Japon\\[10pt]
\noindent Adresse email: wallet.alexandre@gmail.com\\
Page personnelle: {\footnotesize \url{http://awallet.github.io/}}\\

\end{multicols}

\subsection*{Intérêts scientifiques}

\begin{multicols}{2}
  \begin{itemize}
  \item Cryptologie
  \item Algorithmique
  \item Théorie des nombres
  \item Sécurité informatique
  \item Calcul formel
  \item Géométrie algébrique
  \end{itemize}
\end{multicols}
  

\section*{Situation actuelle}

\hspace{-0.6cm} \begin{tabular}{rl}
Depuis~~  & \textbf{Post-doctorant}, NTT Secure Platform Laboratories, Tokyo.\\
                  Février 2019:  & Cryptographie post-quantique, attaque par canaux auxiliaires, réseaux euclidiens (algébriques).
\end{tabular}

\smallskip
\section{Expériences scientifiques et professionnelles}
\hspace{-0.9cm} \begin{tabular}{rl}
                  $01/2017 -12/2018$: & \textbf{Post-doctorant}, Laboratoire de l'Informatique du Parallélisme (LIP), équipe AriC, ENS Lyon\\
                  & Cryptographie post-quantique, réseaux euclidiens, théorie algorithmique des nombres.\\
          &\\
                  $10/2013-12/2016$: & \textbf{Doctorant}, Laboratoire d'Informatique de Paris 6 (LIP6), équipe PolSys, INRIA, UPMC.\\
          & Théorie algorithmique des nombres, cryptographie sur courbes elliptiques.\\
          &\\
                  $09/2012-08/2013$: & \textbf{Enseignant en mathématiques}, Lycée du Parc Chabrière, Oullins (69).\\
          &\\
                 2012: & \textbf{Stage de recherche}, Institut Camille Jordan (ICJ), UCBL, encadré par Denis Perrot.\\
                                   & Mémoire: {\em \'Eléments de K-théorie des $C^*$-algèbres}.\\
          & Algèbres d'opérateurs, géométrie non-commutative.\\
                                 &\\
                 2010: & \textbf{Stage de recherche}, Institut Camille Jordan (ICJ), UCBL, encadré par Christophe Delaunay.\\
                                  & Mémoire: {\em Introduction au problème du logarithme discret}.\\
          & Logarithme discret et cryptographie.
\end{tabular}

\section{Formation}

\hspace{-0.35cm} \begin{tabular}{rl}
 Décembre 2016: & \textbf{Doctorat en informatique}, Université Pierre et Marie Curie, Paris 6.\\
                & Intitulé: {\em Le problème de décomposition de points dans les variétés Jacobiennes.}\\
                & Directeur: Jean-Charles Faugère. \\
                & Encadrante: Vanessa Vitse.\\
                & Rapporteurs: David Lubicz, François Morain. \\
                &Examinateurs: Stef Graillat, Mohab Safey-el-Din.\\
                &  \\
Septembre 2012: & \textbf{Master en mathématiques fondamentales}, \'Ecole Normale Supérieure de Lyon.\\
                % & Mémoire: {\em \'Eléments de K-théorie des $C^*$-algèbres}, encadré par Denis Perrot.\\
                & \\
Juillet 2011:   & \textbf{Agrégation en mathématiques}, préparée à l'Université Claude Bernard, Lyon 1.\\
                & \\
Septembre 2010: & \textbf{Master de mathématiques appliquées}, Université Claude Bernard, Lyon 1.\\
%                 & Mémoire: {\em Introduction au problème du logarithme discret}, encadré par Christophe Delaunay.\\
%                 & \\                  
% Juillet 2007: & \textbf{Licence de mathématiques}, Université Claude Bernard, Lyon 1.\\
%                 & \\
% Juin 2003: & \textbf{Baccalauréat général}, filière scientifique, Lycée Alain Fournier, Bourges.
\end{tabular}

\section{Auditions, classements}

En Mai 2019, j'ai été classé 3e au concours pour un poste de maître de conférence à l'Université de Versailles Saint-Quentin, rattaché à l'équipe de cryptographie du Laboratoire de Mathématiques de Versailles (LMV).


\section{Publications}

Dans mon domaine de recherche, les publications scientifiques sont majoritairement au format d'articles de conférence.
En particulier, toutes les conférences imposent une étape de relecture par les pairs; les plus sélectives sont EUROCRYPT, CRYPTO, ASIACRYPT et TCC.
L'usage est de ranger les auteurs par ordre alphabétique.

\subsection*{Conférences internationales avec comité de sélection}

\begin{itemize}
  
\item[2020:] \href{https://eprint.iacr.org/2019/1456}{\em MODFALCON: compact signatures based on module-NTRU lattices}, C.~Chuengsatiansup, T.~Prest, D.~Stehlé, A.~Wallet, K.~Xagawa, AsiaCCS 2020 \textbf{(à paraître)}.
  
\item[2020:] \href{https://eprint.iacr.org/2019/1180.pdf}{\em Key Recovery from Gram-Schmidt Norm Leakage in Hash-and-Sign Signatures over NTRU Lattices},\\ P.-A.~Fouque, P.~Kirchner, M.~Tibouchi, A.~Wallet, Y.~Yu , EUROCRYPT 2020.
%  \url{https://eprint.iacr.org/2019/1180.pdf}
\item[2019:]  \href{https://eprint.iacr.org/2019/1035.pdf}{\em An LLL algorithm for module lattices}, C.~Lee, A.~Pellet-\hspace{.1pt}-Mary, D.~Stehlé, A.~Wallet, ASIACRYPT 2019.
  Classé dans les trois meilleurs articles de la conférence, invité pour une version étendue dans le {\em Journal of Cryptology}.
 %\url{https://eprint.iacr.org/2019/1035.pdf}

\item[2018:] \href{https://eprint.iacr.org/2018/170.pdf}{\em On the Ring-LWE and Polynomial-LWE problems}, M.~Ro\cs ca, D.~Stehlé, A.~Wallet, EUROCRYPT 2018.
  %\url{https://eprint.iacr.org/2018/170.pdf}
\item[2015:] \href{http://hal.upmc.fr/hal-01203086/document}{\em Improved Sieving over Algebraic Curves}, V.~Vitse, A.~Wallet, LATINCRYPT 2015.
  %\url{http://hal.upmc.fr/hal-01203086/document}
\end{itemize}


\subsection*{Journaux internationaux}

\begin{itemize}
    \item[2019:]   \href{https://eprint.iacr.org/2020/057.pdf}{\em On the smoothing parameter and last minimum of random orthgonal lattices}, E.~Kirshanova, H.~T.~Nguyen, D.~Stehlé, A.~Wallet.
    Designs, Codes and Cryptography (DCC).

  \item[2017:]  \href{https://awallet.github.io/images/HyperSum.pdf}{\em The Point Decomposition Problem in the divisor class group of hyperelliptic curves: toward efficient computations in even characteristic}, J-C.~Faugère, A.~Wallet. Designs, Codes and Cryptography (DCC).
  %\url{https://awallet.github.io/images/HyperSum.pdf}
\end{itemize}
  
\noindent\textbf{A paraître:}
  \begin{itemize}[label=$\bullet$]
  \item   \href{https://eprint.iacr.org/2019/898.pdf}{\em One Bit is All It Takes: a devastating timing attack agains BLISS non constant-time sign flips}, M.~Tibouchi, A.~Wallet. Journal of Mathematical Cryptology.
    \end{itemize}
  

\subsection*{Atelier international, avec comité de sélection}

\begin{itemize}
\item[2019:] \href{https://eprint.iacr.org/2019/898.pdf}{\em One Bit is All It Takes: a devastating attack against BLISS non constant-time sign flips}, M.~Tibouchi, A.~Wallet, MATHCRYPT 2019, affilié à la conférence CRYPTO 2019.
%  \url{https://eprint.iacr.org/2019/898.pdf}
\end{itemize}


\subsection*{Prépublications, rapports, ou  archives}

\begin{itemize}[label=$\bullet$]
\item[2020:] \href{https://eprint.iacr.org/2020/230}{\em Lattice analysis of the MiNTRU problem}, C.~Lee, A.~Wallet, archive ePrint IACR.
\end{itemize}


\section{Collaborations en cours}

\begin{itemize}[label=$\bullet$]

  
  
\item {\em Avec Mehdi Tibouchi (NTT Corporation, Tokyo):}

  Développement d'un utilitaire pour l'analyse du caractère ``temps constant'' des implémentations cryptographiques. Il sera à terme intégré à SUPERCOP, un outil libre permettant de tester le comportement d'algorithmes cryptographiques sous différentes architectures matérielles, options de compilation, optimisation, $\dotsc$  Nous espérons démontrer l'intérêt de notre utilitaire en identifiant, exploitant et corrigeant des fuites dans les implémentations proposées pour les candidats participant à l'appel à standardisation post-quantique du NIST. 
  
% \item {\em Avec C.~Chuengsatiansup (Université d'Adelaide), T.~Prest (PQ Shield), D.~Stehlé (ENS Lyon) et K.~Xagawa (NTT Corporation):}

%   Conception d'un schéma de signature de type {\em Hash-and-sign} instantié à des modules de rang plus élevé que les approches existantes. Le résultat est une famille d'algorithmes efficaces permettant un compromis entre efficacité et sécurité plus fin que le candidat NIST \falcon. Nous fournissons aussi une preuve de sécurité théorique et une analyse de sécurité concrète vis-à-vis de l'état de l'art de la cryptanalyse des schémas réseaux.\\
%   {\bf Article soumis à AsiaCCS.}
  
%\newpage  
% \item{\em Avec T.~Prest (PQ Shield, Oxford) et D.~Stehlé:}

%   Extensions des techniques précédentes à des schémas de signatures ``à la NTRU''. L'obectif est encore une fois d'affiner les paramètres de sécurité, mais aussi de donner plus de souplesse pour l'implantation de certains candidat, comme le schéma FALCON.


\item {\em Avec E.~Kirshanova (Université de Kaliningrad) et A.~Pellet-\hspace{0pt}-Mary (KU Leuven):}

  Travaux généraux sur les réseaux construits à partir de modules sur des anneaux d'entiers algébriques: étude des aspects algorithmiques en présence d'un oracle pour un sous-problème, et cryptanalyse des problèmes de types Learning With Errors formulés pour ces réseaux. Article en cours de rédaction.

    \item \emph{Avec Thomas Espitau (NTT Corporation, Tokyo) :}\\
    Thomas et moi tentons d'unifier les concepts de la théorie cryptographique des réseaux euclidiens à travers la théorie d'Arakelov des courbes arithmétiques telle que développée par Bost, Gillet et Soulé.
    Nous espérons obtenir une théorie plus uniforme et des outils mieux adaptés à l'analyse d'algorithmes récemment présentés pour les modules projectifs.
% \item {\em Avec E.~Kirshanova (ENS Lyon), D.~Stehlé (ENS Lyon):}

%   Encadrement du stage de fin d'étude d'une étudiante du Master informatique de l'ENS Lyon.

\end{itemize}

\paragraph{Projet CNRS-RFFI} Je suis collaborateur externe sur un projet de recherche porté par Elena Kirshanova (Université de Kaliningrad) et proposé pour financement au partenariat entre le CNRS et le Fond Russe pour la Recherche Fondamentale (RFFI).


\section{Diffusions scientifiques et autres activités}

\subsection*{Communications internationales}

\noindent \textbf{Exposé invité:}\\

\hspace{-0.85cm}\begin{tabular}{ll}
24 Avril 2020: & {\em ``Mod-NTRU trapdoors and applications''}, atelier \href{https://simons.berkeley.edu/workshops/lattices-2020-3}{``Lattices: From Theory to Practice''}\\ & Simons Institute for the Theory of Computing, Berkeley, Etats-Unis. (\href{https://www.youtube.com/watch?v=jWrdAFcvzpk}{Vidéo})
                \end{tabular}\\[5pt]

                
\noindent \textbf{Exposés en conférence:}\\[5pt]                
Mes contributions les plus récentes sont partagées avec des doctorants, à qui revient en général l'opportunité de présenter les résultats.
%C'est en effet une partie importante de la formation d'un chercheur que de présenter son travail.
J'ai exposé d'autres mes travaux lors des conférences où ils ont été publiés:\\

\hspace{-0.85cm}\begin{tabular}{ll}
 18 Août 2019: & {\em One Bit is All It Takes: a devastating attack against BLISS' non constant-time sign flips},\\ & MATHCRYPT 2019, Santa Barbara, Etats-Unis.\\[5pt]
 25 Août 2015: & {\em Improved Sieving over Algebraic Curves}, LATINCRYPT 2015, Guadalajara, Mexique.
\end{tabular}


\subsection*{Communications nationales}

J'ai présenté certains de mes résultats de thèse et de post-doctorat sous forme de \textbf{posters} lors de mes participations aux Journées Nationales du groupe de travail d'Informatique-Mathématique (GDR-IM).
J'ai aussi donné \textbf{un court exposé} à l'édition 2017 des Journées Nationales du groupe de travail Codage et Cryptographie (C2).

% \hspace{-0.85cm}\begin{tabular}{ll}
%                   5 Mai 2018: & Poster: {\em On variants of Ring-LWE and Polynomial-LWE problems}.\\
%                   &Journées du GDR-IM, Saclay.\\
%                   24 Avril 2017: & {\em Décompositions de points dans les variétés Jacobiennes hyperelliptiques}.\\
%                   &Journées Codage et Cryptographie (JC2), La Bresse.\\
%                   14 Mars 2017: & Poster: {\em Décompositions de points dans les variétés Jacobiennes hyperelliptiques}.\\
%                   &Journées du GDR-IM, LIRMM, Montpellier.
% \end{tabular}

\subsection*{Séminaires (sélections)}

Nos résultats sur les variantes algébriques de Learning With Errors à EUROCRYPT ont suscité un vif intérêt de la communauté.
J'ai eu l'occasion de le présenter dans \textbf{des exposés longs} à de nombreux séminaires en France et à l'étranger.
La liste ci-dessous recense certains de ces exposés. D'autres exposés longs plus anciens concernent mes travaux de thèse.\\

\noindent\textbf{Internationaux:}\\

\hspace{-0.85cm}
\begin{tabular}{ll}
  
  26 Août 2019: &  %{\em On variants of Ring-LWE and Polynomial-LWE}.\\
  % &
      INTEL, San Diego.\\[5pt]
  
  16 Janvier 2019: & % {\em On variants of Ring-LWE and Polynomial-LWE}.\\
  Cryptography and security seminar, PQ Shield, Mathematical Institute of Oxford University.\\[5pt]
  
  11 Septembre 2018: &  %{\em On variants of Ring-LWE and Polynomial-LWE}.\\
                    Cryptography seminar, NTT Secure Platform Laboratories, Tokyo.
\end{tabular}

\bigskip
\noindent\textbf{Nationaux:}\\

\hspace{-0.85cm}\begin{tabular}{ll}
  15 Juin 2018: & %{\em On variants of Ring-LWE and Polynomial-LWE}, exposé en anglais.\\
                  Séminaire CCA de l'INRIA, Centre INRIA de Paris.\\[5pt]
  20 Octobre 2017: &  %{\em On variants of Ring-LWE and Polynomial-LWE}, exposé en anglais.\\
                  Lattice meetings, ENS Lyon.\\[5pt]
  17 Mai 2017: &  %{\em Décompositions de points dans les variétés Jacobiennes}.\\
                  Séminaire de l'équipe ECO/ESCAPE, LIRMM, Montpellier.\\[5pt]
  8 Décembre 2016:~~~\, &  %{\em Décompositions de points dans les variétés Jacobiennes}.\\
                  Séminaire GTBAC, Télécom ParisTech.
\end{tabular}

%\newpage
\subsubsection*{Activités administratives et d'organisation}

\begin{itemize}[label = $\bullet$]
  \item Automne 2018: pendant mon post-doctorat à l'ENS de Lyon, j'ai participé à l'organisation de l'édition 2018 des \href{https://jc2-2018.sciencesconf.org/}{Journées C2}.
\item Juillet 2017: j'ai fait partie du comité d'harmonisation pour l'obtention du label \href{https://euraxess.ec.europa.eu/jobs/hrs4r}{HRS4R} (Human Ressource Strategies For Researchers) à l'ENS de Lyon, en tant que représentant des doctorants et post-doctorants. %J'ai participé aux échanges des réunions en partageant mon expérience.
\item Septembre 2015: j'ai été membre du comité d'organisation de la conférence internationale Cryptographic Hardware and Embedded Systems (\href{https://ches.iacr.org/2015/}{CHES}), à Saint-Malo. %Mon travail s'est partagé entre l'accueil et la réception des participants, la gestion des problèmes ``au fil de l'eau'', et l'organisation des transitions entre les exposés et les séances.
\end{itemize}

\subsubsection*{Comités de relectures}

\begin{itemize}[label = $\bullet$]
\item Printemps 2020: membre du comité de programme de la conférence bi-annuelle de théorie algorithmique des nombres \href{https://www.math.auckland.ac.nz/~sgal018/ANTS/}{ANTS}.
\end{itemize}
La communauté publiant essentiellement en conférence, une activité habituelle des chercheurs du domaine concerne l'évaluation des soumissions.
Je suis régulièrement relecteur externe pour les conférences de l'IACR, particulièrement CRYPTO, EUROCRYPT, et ASIACRYPT.
J'ai participé plus sporadiquement au processus pour d'autres conférences du domaine comme TCC, ainsi que pour la conférence bi-annuelle de théorie algorithmique des nombres ANTS.

% J'ai été (ou suis actuellement) relecteur pour les conférences suivantes:
% \begin{multicols}{2}
% \begin{itemize}
% \item EUROCRYPT depuis 2017
% \item CRYPTO 2018 et 2019
% \item ASIACRYPT 2017, 2018, 2019
% \item TCC 2019
% \item ANTS XIII
% \item INDOCRYPT 2018
% \item AFRICACRYPT 2017
% \item PKC 2015
% \end{itemize}
% \end{multicols}

\section{Encadrements scientifiques, enseignements}

% \begin{itemize}[label=$\bullet$]
% \item
  \paragraph{Encadrement d'un stage de recherche:} {\em Huyen Nguyen}, étudiante du M2 informatique de l'ENS de Lyon, promotion 2018. Co-encadrée avec Elena Kirshanova (ENS Lyon) et Damien Stehlé (ENS Lyon):\\[5pt]
    Sujet du stage : Etude des quantités fondamentales (minima successifs, paramètre de lissage) de réseaux orthogonaux à des matrices dont les entrées ont été tirées selon une distribution Gaussienne discrète. Cet encadrement a conduit à l'écriture d'un article en publication au journal {\em Designs, Codes and Cryptography}. Huyen poursuit actuellement son travail en thèse.

    \paragraph{Encadrement scientifique non officiel:}

    Cet encart correspond à des activités et responsabilités qui n'ont pas été pas clairement établies par des documents administratifs, mais dépassant nettement le cadre de discussions scientifiques épisodiques.
  
    \begin{itemize}
    \item  {\em Miruna Ro\cs ca}, doctorante sous la direction de Damien Stehlé (ENS Lyon), en cotutelle avec BitDefender (Bucarest, Roumanie).
%      Pendant mon post-doctorat à Lyon, j'ai participé à l'encadrement scientifique de Miruna lors de ses séjours à Lyon.
    \item {\em Emily Clément}, étudiante de master à Rennes, encadrée par Adeline Roux-Langlois (IRISA).
%      J'ai orienté Emily dans son travail autour des résultats parus à EUROCRYPT 2018, ainsi que dans la rédaction et la présentation de son mémoire. 
    \end{itemize}
    
\newpage
\subsection*{Activités d'enseignement}

J'ai six années d'activité en tant qu'enseignant, partagée entre le secondaire et le supérieur, et les mathématiques et l'informatique. En post-doctorat, j'ai effectué 12 heures d'enseignement en première année du master d'informatique, et participé à l'évaluation des stages des étudiants de licence à l'ENS de Lyon. Pendant mes trois années de thèse, j'ai effectué 192 heures d'enseignement d'informatique en licence à l'UPMC (campus Jussieu), sur des thèmes variés. Avant ma thèse, j'ai eu la responsabilité de deux classes de secondes à temps plein pendant une année, en tant qu'enseignant agrégé de mathématiques. Ceci correspond essentiellement à un volume de 12 heures hebdomadaires sur 36 semaines, soit 432 heures (sans inclure les heures de formation pour les enseignants stagiaires).
La Table~\ref{tab:ens} résume ce paragraphe.

\vspace{\baselineskip}
%\noindent La Table \ref{tab:ens} ci-dessous résume mes activités de manière concise. % Plus de détails sur le contenu des enseignements et mes responsabilités sont donnés en Annexe~\ref{appA:ens}.
\begin{table}[h]
  
  \centering
  \caption{{\bf Résumé des activités antérieures d'enseignement}}
  \label{tab:ens}
  \vspace{0.75\baselineskip}
  \renewcommand{\arraystretch}{1.2}
  \resizebox{0.9\textwidth}{!}{
    \begin{tabular}{lcccc}
      \toprule
  Type & Niveau & Fonction & Volume horaire & Effectif \\
       &        &          & (équivalent TD)&           \\ \midrule
      Encadrement d'un stage & M2 & - & - & 1 \\\rowcolor{black!10}
      Calcul Formel & M1 & TD & 12 heures & $\sim$ 15 \\
  %\rule{0pt}{0.5cm}
  Introduction à la cryptologie & L3 & TD/TP & 63 heures & $\sim$ 25 \\\rowcolor{black!10}
   % \rule{0pt}{0.5cm}
  Calcul scientifique & L2 & TP & 20 heures & $\sim$ 20 \\
  %\rule{0pt}{0.5cm}
  Types et structures de données en C & L2 & TP & 18 heures & $\sim$ 20\\\rowcolor{black!10}
  %\rule{0pt}{0.5cm}
  Architecture machine et représentation & L2 & TP & 21 heures & $\sim$ 20\\
  %\rule{0pt}{0.5cm}
  Environnement de développement & \multirow{2}{*}{L2}  & \multirow{2}{*}{TP} & \multirow{2}{*}{30 heures} & \multirow{2}{*}{$\sim$ 25}\\
    et compilation & & & & \\\rowcolor{black!10}
  %\rule{0pt}{0.5cm}
  Structures discrètes & L2 & TP & 21 heures & $\sim$ 15\\
  %\rule{0pt}{0.5cm}
  Initiation à la programmation& \multirow{2}{*}{L1} & \multirow{2}{*}{TP} & \multirow{2}{*}{19 heures} & \multirow{2}{*}{$\sim$ 25}\\
          avec Python & & & & \\\rowcolor{black!10}
   %       \rule{0pt}{0.5cm}
      Enseignant de mathématiques & 2nde & - & 432 heures (1 an) & $\sim$ 35  \\
      Formation enseignant (IUFM) & - & - & $\sim$ 168 heures & - \\\rowcolor{black!10}
    %        \rule{0pt}{0.5cm}
     %       Khôlles de mathématiques & spé TSI ($\sim$ L2) & - & $\sim$ 40 heures & 3 par Khôlle\\
\bottomrule

\end{tabular}
}
\end{table}


\end{document}

%%% Local Variables:
%%% mode: latex
%%% TeX-master: t
%%% End:
